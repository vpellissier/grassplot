\documentclass[table]{article}
\usepackage{lmodern}
\usepackage{amssymb,amsmath}
\usepackage{ifxetex,ifluatex}
\usepackage{fixltx2e} % provides \textsubscript
\ifnum 0\ifxetex 1\fi\ifluatex 1\fi=0 % if pdftex
  \usepackage[T1]{fontenc}
  \usepackage[utf8]{inputenc}
\else % if luatex or xelatex
  \ifxetex
    \usepackage{mathspec}
  \else
    \usepackage{fontspec}
  \fi
  \defaultfontfeatures{Ligatures=TeX,Scale=MatchLowercase}
\fi
% use upquote if available, for straight quotes in verbatim environments
\IfFileExists{upquote.sty}{\usepackage{upquote}}{}
% use microtype if available
\IfFileExists{microtype.sty}{%
\usepackage{microtype}
\UseMicrotypeSet[protrusion]{basicmath} % disable protrusion for tt fonts
}{}
\usepackage[margin=1in]{geometry}
\usepackage{hyperref}
\hypersetup{unicode=true,
            pdftitle={Report land-use data},
            pdfauthor={Vincent Pellissier},
            pdfborder={0 0 0},
            breaklinks=true}
\urlstyle{same}  % don't use monospace font for urls
\usepackage{color}
\usepackage{fancyvrb}
\newcommand{\VerbBar}{|}
\newcommand{\VERB}{\Verb[commandchars=\\\{\}]}
\DefineVerbatimEnvironment{Highlighting}{Verbatim}{commandchars=\\\{\}}
% Add ',fontsize=\small' for more characters per line
\usepackage{framed}
\definecolor{shadecolor}{RGB}{248,248,248}
\newenvironment{Shaded}{\begin{snugshade}}{\end{snugshade}}
\newcommand{\KeywordTok}[1]{\textcolor[rgb]{0.13,0.29,0.53}{\textbf{#1}}}
\newcommand{\DataTypeTok}[1]{\textcolor[rgb]{0.13,0.29,0.53}{#1}}
\newcommand{\DecValTok}[1]{\textcolor[rgb]{0.00,0.00,0.81}{#1}}
\newcommand{\BaseNTok}[1]{\textcolor[rgb]{0.00,0.00,0.81}{#1}}
\newcommand{\FloatTok}[1]{\textcolor[rgb]{0.00,0.00,0.81}{#1}}
\newcommand{\ConstantTok}[1]{\textcolor[rgb]{0.00,0.00,0.00}{#1}}
\newcommand{\CharTok}[1]{\textcolor[rgb]{0.31,0.60,0.02}{#1}}
\newcommand{\SpecialCharTok}[1]{\textcolor[rgb]{0.00,0.00,0.00}{#1}}
\newcommand{\StringTok}[1]{\textcolor[rgb]{0.31,0.60,0.02}{#1}}
\newcommand{\VerbatimStringTok}[1]{\textcolor[rgb]{0.31,0.60,0.02}{#1}}
\newcommand{\SpecialStringTok}[1]{\textcolor[rgb]{0.31,0.60,0.02}{#1}}
\newcommand{\ImportTok}[1]{#1}
\newcommand{\CommentTok}[1]{\textcolor[rgb]{0.56,0.35,0.01}{\textit{#1}}}
\newcommand{\DocumentationTok}[1]{\textcolor[rgb]{0.56,0.35,0.01}{\textbf{\textit{#1}}}}
\newcommand{\AnnotationTok}[1]{\textcolor[rgb]{0.56,0.35,0.01}{\textbf{\textit{#1}}}}
\newcommand{\CommentVarTok}[1]{\textcolor[rgb]{0.56,0.35,0.01}{\textbf{\textit{#1}}}}
\newcommand{\OtherTok}[1]{\textcolor[rgb]{0.56,0.35,0.01}{#1}}
\newcommand{\FunctionTok}[1]{\textcolor[rgb]{0.00,0.00,0.00}{#1}}
\newcommand{\VariableTok}[1]{\textcolor[rgb]{0.00,0.00,0.00}{#1}}
\newcommand{\ControlFlowTok}[1]{\textcolor[rgb]{0.13,0.29,0.53}{\textbf{#1}}}
\newcommand{\OperatorTok}[1]{\textcolor[rgb]{0.81,0.36,0.00}{\textbf{#1}}}
\newcommand{\BuiltInTok}[1]{#1}
\newcommand{\ExtensionTok}[1]{#1}
\newcommand{\PreprocessorTok}[1]{\textcolor[rgb]{0.56,0.35,0.01}{\textit{#1}}}
\newcommand{\AttributeTok}[1]{\textcolor[rgb]{0.77,0.63,0.00}{#1}}
\newcommand{\RegionMarkerTok}[1]{#1}
\newcommand{\InformationTok}[1]{\textcolor[rgb]{0.56,0.35,0.01}{\textbf{\textit{#1}}}}
\newcommand{\WarningTok}[1]{\textcolor[rgb]{0.56,0.35,0.01}{\textbf{\textit{#1}}}}
\newcommand{\AlertTok}[1]{\textcolor[rgb]{0.94,0.16,0.16}{#1}}
\newcommand{\ErrorTok}[1]{\textcolor[rgb]{0.64,0.00,0.00}{\textbf{#1}}}
\newcommand{\NormalTok}[1]{#1}
\usepackage{graphicx,grffile}
\makeatletter
\def\maxwidth{\ifdim\Gin@nat@width>\linewidth\linewidth\else\Gin@nat@width\fi}
\def\maxheight{\ifdim\Gin@nat@height>\textheight\textheight\else\Gin@nat@height\fi}
\makeatother
% Scale images if necessary, so that they will not overflow the page
% margins by default, and it is still possible to overwrite the defaults
% using explicit options in \includegraphics[width, height, ...]{}
\setkeys{Gin}{width=\maxwidth,height=\maxheight,keepaspectratio}
\IfFileExists{parskip.sty}{%
\usepackage{parskip}
}{% else
\setlength{\parindent}{0pt}
\setlength{\parskip}{6pt plus 2pt minus 1pt}
}
\setlength{\emergencystretch}{3em}  % prevent overfull lines
\providecommand{\tightlist}{%
  \setlength{\itemsep}{0pt}\setlength{\parskip}{0pt}}
\setcounter{secnumdepth}{0}
% Redefines (sub)paragraphs to behave more like sections
\ifx\paragraph\undefined\else
\let\oldparagraph\paragraph
\renewcommand{\paragraph}[1]{\oldparagraph{#1}\mbox{}}
\fi
\ifx\subparagraph\undefined\else
\let\oldsubparagraph\subparagraph
\renewcommand{\subparagraph}[1]{\oldsubparagraph{#1}\mbox{}}
\fi

%%% Use protect on footnotes to avoid problems with footnotes in titles
\let\rmarkdownfootnote\footnote%
\def\footnote{\protect\rmarkdownfootnote}

%%% Change title format to be more compact
\usepackage{titling}

% Create subtitle command for use in maketitle
\newcommand{\subtitle}[1]{
  \posttitle{
    \begin{center}\large#1\end{center}
    }
}

\setlength{\droptitle}{-2em}

  \title{Report land-use data}
    \pretitle{\vspace{\droptitle}\centering\huge}
  \posttitle{\par}
    \author{Vincent Pellissier}
    \preauthor{\centering\large\emph}
  \postauthor{\par}
      \predate{\centering\large\emph}
  \postdate{\par}
    \date{5 November 2018}


\begin{document}
\maketitle

This report provides information on the structure of the land-use
information present in the GrassPlot dataset, along with the code used.

\subsection{Dataset}\label{dataset}

The master file (Grassplot 1.8.xlsx as of Nov. 5th 2018) contains five
sheets:

\begin{itemize}
\tightlist
\item
  log book: tracking of the modifcations of the masterfile
\item
  Data: list of plot data (part of the datasets) contributed to
  Grassplot, with info on the plot data (land-use data, environmnental
  data)
\item
  datasets: list of the datasets contributed to Grassplot, with metadata
  on the datasets (nested or not, are compositional data present, are
  there any environmental data\ldots{})
\item
  new references: new (?) references added to the data
\item
  Consortium members: list of the consortium members along with info
\end{itemize}

The sheet containing land-use information at the plot level is the
`Data' sheet. As the last 11 columns (with LU information) have coding
error (ex. decimal separator, characters in a numerical column,
\ldots{}), these columns are read as text columns and will be corrected
later. Note that when loading columns with a comma as separator as a
text column, the comma is replaced by a period (for an unknown
reason\ldots{})

\begin{Shaded}
\begin{Highlighting}[]
\CommentTok{# dataframe with all the informations at a plot level}
\NormalTok{df <-}\StringTok{ }\KeywordTok{read_excel}\NormalTok{(}\KeywordTok{file.path}\NormalTok{(path_grassplot, }\StringTok{"GrassPlot 1.8.xlsx"}\NormalTok{ ),}
                 \DataTypeTok{sheet =} \StringTok{'Data'}\NormalTok{, }
                 \DataTypeTok{col_types =} \KeywordTok{rep}\NormalTok{(}\StringTok{'guess'}\NormalTok{, }\DecValTok{94}\NormalTok{), }\KeywordTok{rep}\NormalTok{(}\StringTok{'text'}\NormalTok{, }\DecValTok{11}\NormalTok{))}
\KeywordTok{dim}\NormalTok{(df)}
\end{Highlighting}
\end{Shaded}

\begin{verbatim}
## [1] 180077    104
\end{verbatim}

\subsection{Land-use information}\label{land-use-information}

\subsubsection{Overview of the information
available}\label{overview-of-the-information-available}

There are 11 columns containing potential information about land-use:

\begin{verbatim}
##  [1] "Land use (5 standard categories: mown, grazed, abandoned, natural grassland, NA)"                                                        
##  [2] "Land use detail (e.g. number of cuts or years since abandonment)"                                                                        
##  [3] "Mowing (1/0)"                                                                                                                            
##  [4] "Mowing frequency: cuts per year (2=2cut/yr; 1=1cut/yr; 0.5=1 cut/2 yr or 2 yr abandoned; 0.2=1 cut/5 yr or 5 yr abandones; 0=never mown)"
##  [5] "Grazing (1/0)"                                                                                                                           
##  [6] "Grazing intensity (0/1) 1= intensive grazing"                                                                                            
##  [7] "Burning (1/0)"                                                                                                                           
##  [8] "Ex arable (1/0)"                                                                                                                         
##  [9] "Ex arable years (years since last ploughing)"                                                                                            
## [10] "Fertilized (1/0)"
\end{verbatim}

Some of these columns contains hardcoded NAs (\emph{i.e.} as character
chain, `NA' or `{[}NA{]}') which can be replaced by proper NAs:

\begin{Shaded}
\begin{Highlighting}[]
\NormalTok{df <-}\StringTok{ }\NormalTok{df }\OperatorTok
\StringTok{        }\KeywordTok{mutate_at}\NormalTok{(}\DataTypeTok{.vars =} \KeywordTok{c}\NormalTok{(}\DecValTok{95}\OperatorTok{:}\DecValTok{104}\NormalTok{), }\KeywordTok{funs}\NormalTok{(}\KeywordTok{ifelse}\NormalTok{(. }\OperatorTok\StringTok{ }\KeywordTok{c}\NormalTok{(}\StringTok{'NA'}\NormalTok{, }\StringTok{'[NA]'}\NormalTok{), }\OtherTok{NA}\NormalTok{, .)))}
\end{Highlighting}
\end{Shaded}

These columns are supposidely having information coded in a varying
number of values, repectively 5, undetermined number, 2, 5, 2, 2, 2, 2,
undertermined number, 2. In reality, the columns contain the following
number of values:

\begin{tabular}{>{\raggedright\arraybackslash}p{8cm}|>{\raggedright\arraybackslash}p{3cm}|>{\raggedright\arraybackslash}p{3cm}}
\hline
  & Expected number of values & Actual number of values\\
\hline
Land use (5 standard categories: mown, grazed, abandoned, natural grassland, NA) & 5 & 32\\
\hline
Land use detail (e.g. number of cuts or years since abandonment) & NA & 917\\
\hline
Mowing (1/0) & 2 & 3\\
\hline
Mowing frequency: cuts per year (2=2cut/yr; 1=1cut/yr; 0.5=1 cut/2 yr or 2 yr abandoned; 0.2=1 cut/5 yr or 5 yr abandones; 0=never mown) & 5 & 9\\
\hline
Grazing (1/0) & 2 & 5\\
\hline
Grazing intensity (0/1) 1= intensive grazing & 2 & 14\\
\hline
Burning (1/0) & 2 & 3\\
\hline
Ex arable (1/0) & 2 & 2\\
\hline
Ex arable years (years since last ploughing) & NA & 5\\
\hline
Fertilized (1/0) & 2 & 4\\
\hline
\end{tabular}

\subsubsection{Details on the land-use information
available}\label{details-on-the-land-use-information-available}

\paragraph{Land use (column 95 / column
CQ)}\label{land-use-column-95-column-cq}

\subparagraph{\texorpdfstring{\emph{Problems}}{Problems}}\label{problems}

This column contains 32 values instead of 5:

\begin{verbatim}
## $`Land use (5 standard categories: mown, grazed, abandoned, natural grassland, NA)`
##  [1] "grazed"                    "grazed, mown"             
##  [3] "mown"                      "natural grassland"        
##  [5] "abandoned"                 NA                         
##  [7] "grazed/natural"            "grazed/mown"              
##  [9] "grazing"                   "Abandoned"                
## [11] "Mown and grazed"           "Mown"                     
## [13] "mown and grazed"           "Natural grassland"        
## [15] "mown/abandoned"            "grazed/abandoned"         
## [17] "mown/grazed"               "abandoned?"               
## [19] "LAS"                       "P"                        
## [21] "L"                         "N"                        
## [23] "grazedazed"                "grazed, burnt"            
## [25] "burnt, abandoned"          "abandonment"              
## [27] "abondonment"               "0 intensive grazing"      
## [29] "0t visible"                "natural grassland, grazed"
## [31] "grazed and mown"           "trampled"                 
## [33] "irregularly mown"
\end{verbatim}

\begin{itemize}
\tightlist
\item
  Some of the values are just typos (\emph{e.g.} capital letter in the
  beginning, misspelling, extra verbose\ldots{}).
\item
  Some are actually two or more LU classes combined (\emph{e.g.}
  grazed/natural\ldots{})
\item
  Some are LU classes not listed (\emph{e.g.} trampled, grazed/burnt)
\item
  Some seems to indicate the absence of LU (\emph{e.g.} 0t visible)
\item
  Some are not comprehensible (\emph{e.g.} LAS, P, L, N)
\end{itemize}

\subparagraph{\texorpdfstring{\emph{Suggestions}}{Suggestions}}\label{suggestions}

\begin{itemize}
\tightlist
\item
  Fixing the typos
\item
  Deciding whether it is better to have mixed classes, or 5 columns
  coded as binary variables. If the former is better, then one should
  homogeneize the way the mixed classed are coded. I suggest to stick to
  the original order, with different LU separated by a `/'. It is
  noteworthy that these binary columns already somehow exists
  (\emph{i.e} there is already Grazing, Mowing, Ex-arable (= abandoned?)
  binary columns). See below for more details on these columns.
\item
  Either adding classes to matches the new LU classes, matching with the
  closest class, or removing completely, providing there is no more info
  in Column 96 (LU details).
\item
  Replacing by NA, are there is no LU visible
\end{itemize}

\paragraph{Land use details (column 96 / column
CR)}\label{land-use-details-column-96-column-cr}

\subparagraph{\texorpdfstring{\emph{Problems}}{Problems}}\label{problems-1}

This columns contain 917 unique values, with various information, freely
written by data provider. The information is not usable as such, as it
is not standardized, for instance:

\begin{verbatim}
## [1] "very extensive grazing"                     
## [2] NA                                           
## [3] "mowing once a year at the end of July"      
## [4] "Extensive grazing by goats"                 
## [5] "Low intensity horse pasture, montane meadow"
## [6] "Low intensity pasture (cattle, horses)"
\end{verbatim}

\subparagraph{\texorpdfstring{\emph{Suggestions}}{Suggestions}}\label{suggestions-1}

Manually create a lookup table to match this columns and the other. One
should try to extract intensity information when possible.

\paragraph{Mowing 0/1 (column 97 / column
CS)}\label{mowing-01-column-97-column-cs}

\subparagraph{\texorpdfstring{\emph{Problems}}{Problems}}\label{problems-2}

\begin{itemize}
\tightlist
\item
  Some plots have impossible values (`?')
\item
  Some plots have mown = 1 but are classified as abandoned or natural
  grassland in column 95
\item
  Some plots with mown = 0 are classified as mown in column 95:
\item
  Some plots with mown = NA are classified as mown in column 95
\end{itemize}

\begin{tabular}{>{\raggedright\arraybackslash}p{8cm}|r|r|r|r}
\hline
Land use (5 standard categories: mown, grazed, abandoned, natural grassland, NA) & ? & 0 & 1 & <NA>\\
\hline
0 intensive grazing & NA & NA & NA & 1\\
\hline
0t visible & NA & NA & NA & 2\\
\hline
abandoned & NA & 858 & 12 & 15194\\
\hline
Abandoned & NA & NA & NA & 67\\
\hline
abandoned? & NA & 6 & NA & NA\\
\hline
abandonment & NA & NA & NA & 1\\
\hline
abondonment & NA & NA & NA & 4\\
\hline
burnt, abandoned & NA & 13 & NA & NA\\
\hline
grazed & NA & 3794 & NA & 30153\\
\hline
grazed and mown & NA & NA & 4 & NA\\
\hline
grazed, burnt & NA & NA & NA & 26\\
\hline
grazed, mown & NA & NA & 128 & 12\\
\hline
grazed/abandoned & NA & NA & NA & 23\\
\hline
grazed/mown & NA & NA & NA & 15\\
\hline
grazed/natural & NA & NA & NA & 64\\
\hline
grazedazed & NA & NA & NA & 1\\
\hline
grazing & NA & NA & NA & 10\\
\hline
irregularly mown & NA & NA & 1 & NA\\
\hline
L & NA & NA & NA & 12\\
\hline
LAS & NA & NA & NA & 12\\
\hline
mown & NA & 4 & 430 & 6578\\
\hline
Mown & NA & NA & 34 & 2\\
\hline
mown and grazed & NA & NA & NA & 178\\
\hline
Mown and grazed & NA & NA & NA & 15\\
\hline
mown/abandoned & NA & NA & NA & 10\\
\hline
mown/grazed & NA & NA & 7 & NA\\
\hline
N & NA & NA & NA & 28\\
\hline
natural grassland & NA & 1562 & 4 & 104572\\
\hline
Natural grassland & NA & 17 & NA & NA\\
\hline
natural grassland, grazed & NA & NA & NA & 78\\
\hline
P & NA & NA & NA & 48\\
\hline
trampled & NA & 1 & NA & NA\\
\hline
NA & 6 & 22 & NA & 16068\\
\hline
\end{tabular}

\subparagraph{\texorpdfstring{\emph{Suggestions}}{Suggestions}}\label{suggestions-2}

\begin{itemize}
\tightlist
\item
  Deciding which column takes precedence (column 95 or binary column),
  and reclassify accordingly. To do so, plots (or datasets) with
  discrepancies need to be identified and the original tables need to be
  manually checked.
\item
  Plots that are not mown should be identified with mown = 0 (\emph{i.e}
  not NAs in this column)
\end{itemize}

\paragraph{Mowing frequency (column 98 / column
CT)}\label{mowing-frequency-column-98-column-ct}

\subparagraph{\texorpdfstring{\emph{Problems}}{Problems}}\label{problems-3}

Some values (4, 0.3, 0.03, 0.05, 0.1) do not match the values that
should be present. In addition, some plots listed a mown = 0 have a
mowing frequency:

\begin{tabular}{>{\raggedright\arraybackslash}p{8cm}|r|r|r|r}
\hline
Mowing frequency: cuts per year (2=2cut/yr; 1=1cut/yr; 0.5=1 cut/2 yr or 2 yr abandoned; 0.2=1 cut/5 yr or 5 yr abandones; 0=never mown) & ? & 0 & 1 & <NA>\\
\hline
0 & NA & 468 & NA & NA\\
\hline
0.03 & NA & 16 & NA & NA\\
\hline
0.05 & NA & 159 & NA & NA\\
\hline
0.1 & NA & 30 & NA & NA\\
\hline
0.2 & NA & NA & 4 & NA\\
\hline
0.3 & NA & 4 & 8 & NA\\
\hline
0.5 & NA & NA & 31 & NA\\
\hline
1 & NA & 4 & 212 & NA\\
\hline
4 & NA & NA & 15 & NA\\
\hline
NA & 6 & 5596 & 350 & 173174\\
\hline
\end{tabular}

\subparagraph{\texorpdfstring{\emph{Suggestions}}{Suggestions}}\label{suggestions-3}

\begin{itemize}
\tightlist
\item
  Checking in the column 96 if the frequency information can be
  corrected
\item
  Reclassifying plots with a mowing freaquency as mown = 1 (\emph{i.e.}
  the frequency information takes precedences over the binary info. This
  rule should stand for all the frequency information)
\item
  If no more info can be found, checking with Idoia
\end{itemize}

\paragraph{Grazing (1/0) (column 99 / column
CU)}\label{grazing-10-column-99-column-cu}

\subparagraph{\texorpdfstring{\emph{Problems}}{Problems}}\label{problems-4}

\begin{itemize}
\tightlist
\item
  Some plots have impossible values (`?', `probably' or `2')
\item
  Some plots with grazing = 1 are listed as abandoned, mown or natural
  grassland
\item
  Some plots with grazing = NA are listed as grazed or mixed grazed
\end{itemize}

\begin{tabular}{>{\raggedright\arraybackslash}p{8cm}|r|r|r|r|r|r}
\hline
Land use (5 standard categories: mown, grazed, abandoned, natural grassland, NA) & ? & 0 & 1 & 2 & probably & <NA>\\
\hline
0 intensive grazing & NA & NA & NA & NA & NA & 1\\
\hline
0t visible & NA & NA & NA & NA & NA & 2\\
\hline
abandoned & NA & 1645 & 38 & NA & NA & 14381\\
\hline
Abandoned & NA & NA & NA & NA & NA & 67\\
\hline
abandoned? & NA & 6 & NA & NA & NA & NA\\
\hline
abandonment & NA & NA & NA & NA & NA & 1\\
\hline
abondonment & NA & NA & NA & NA & NA & 4\\
\hline
burnt, abandoned & NA & NA & NA & NA & NA & 13\\
\hline
grazed & NA & NA & 5103 & 1 & NA & 28843\\
\hline
grazed and mown & NA & NA & NA & NA & NA & 4\\
\hline
grazed, burnt & NA & NA & 26 & NA & NA & NA\\
\hline
grazed, mown & NA & NA & 130 & NA & NA & 10\\
\hline
grazed/abandoned & NA & NA & NA & NA & NA & 23\\
\hline
grazed/mown & NA & NA & NA & NA & NA & 15\\
\hline
grazed/natural & NA & NA & NA & NA & NA & 64\\
\hline
grazedazed & NA & NA & NA & NA & NA & 1\\
\hline
grazing & NA & NA & NA & NA & NA & 10\\
\hline
irregularly mown & NA & NA & NA & NA & NA & 1\\
\hline
L & NA & NA & NA & NA & NA & 12\\
\hline
LAS & NA & NA & NA & NA & NA & 12\\
\hline
mown & NA & 176 & 39 & NA & 6 & 6791\\
\hline
Mown & NA & NA & 34 & NA & NA & 2\\
\hline
mown and grazed & NA & NA & NA & NA & NA & 178\\
\hline
Mown and grazed & NA & NA & NA & NA & NA & 15\\
\hline
mown/abandoned & NA & NA & NA & NA & NA & 10\\
\hline
mown/grazed & 7 & NA & NA & NA & NA & NA\\
\hline
N & NA & 28 & NA & NA & NA & NA\\
\hline
natural grassland & NA & 1349 & 60 & NA & NA & 104729\\
\hline
Natural grassland & NA & 17 & NA & NA & NA & NA\\
\hline
natural grassland, grazed & NA & NA & NA & NA & NA & 78\\
\hline
P & NA & NA & NA & NA & NA & 48\\
\hline
trampled & NA & NA & NA & NA & NA & 1\\
\hline
NA & 6 & 17 & NA & NA & NA & 16073\\
\hline
\end{tabular}

\subparagraph{\texorpdfstring{\emph{Suggestions}}{Suggestions}}\label{suggestions-4}

\begin{itemize}
\tightlist
\item
  Decide which column takes precedence (column 95 or binary column), and
  reclassify accordingly. I have no strong opinion on that, we should
  check which information was provided first
\item
  Plots that are not grazed should be identified with grazed = 0
  (\emph{i.e} not NAs in this column)
\end{itemize}

\paragraph{Grazing intensity}\label{grazing-intensity}

\subparagraph{\texorpdfstring{\emph{Problems}}{Problems}}\label{problems-5}

\begin{itemize}
\tightlist
\item
  We have no information on the values that should be in this column
\item
  Some information are coded as plain text
\item
  Some plots with grazing = 1 have no grazing frequency:
\end{itemize}

\begin{tabular}{>{\raggedright\arraybackslash}p{8cm}|r|r|r|r|r|r}
\hline
Grazing intensity (0/1) 1= intensive grazing & ? & 0 & 1 & 2 & probably & <NA>\\
\hline
0 & NA & 1611 & 219 & NA & NA & NA\\
\hline
0.1 & NA & NA & 202 & NA & NA & NA\\
\hline
0.25 & NA & NA & 448 & NA & NA & NA\\
\hline
0.3 & NA & NA & 92 & NA & NA & NA\\
\hline
0.5 & NA & NA & 749 & NA & NA & 8\\
\hline
0.75 & NA & NA & 148 & NA & NA & NA\\
\hline
1 & NA & NA & 2961 & NA & NA & NA\\
\hline
2 & NA & NA & 28 & NA & NA & NA\\
\hline
3 & NA & NA & 31 & NA & NA & NA\\
\hline
7 & NA & NA & 4 & NA & NA & NA\\
\hline
high & NA & NA & 12 & NA & NA & NA\\
\hline
low & NA & NA & 15 & NA & NA & NA\\
\hline
middle & NA & NA & 30 & NA & NA & NA\\
\hline
overgrazing & NA & NA & 3 & NA & NA & NA\\
\hline
NA & 13 & 1627 & 488 & 1 & 6 & 171381\\
\hline
\end{tabular}

\subparagraph{\texorpdfstring{\emph{Suggestions}}{Suggestions}}\label{suggestions-5}

\begin{itemize}
\tightlist
\item
  Checking what values should be there with Idoia
\end{itemize}

\paragraph{Burning (1/0) (column 101 / column
CW)}\label{burning-10-column-101-column-cw}

\subparagraph{\texorpdfstring{\emph{Problems}}{Problems}}\label{problems-6}

\begin{itemize}
\tightlist
\item
  Plots classified as burnt (grazed, burnt and movn, burnt) have burn =
  NA
\end{itemize}

\begin{tabular}{>{\raggedright\arraybackslash}p{8cm}|r|r|r|r}
\hline
Land use (5 standard categories: mown, grazed, abandoned, natural grassland, NA) & ? & 0 & 1 & <NA>\\
\hline
0 intensive grazing & NA & NA & NA & 1\\
\hline
0t visible & NA & NA & NA & 2\\
\hline
abandoned & NA & 658 & 29 & 15377\\
\hline
Abandoned & NA & NA & NA & 67\\
\hline
abandoned? & 6 & NA & NA & NA\\
\hline
abandonment & NA & NA & NA & 1\\
\hline
abondonment & NA & NA & NA & 4\\
\hline
burnt, abandoned & NA & NA & NA & 13\\
\hline
grazed & NA & 3649 & 52 & 30246\\
\hline
grazed and mown & NA & 4 & NA & NA\\
\hline
grazed, burnt & NA & NA & NA & 26\\
\hline
grazed, mown & NA & 7 & 14 & 119\\
\hline
grazed/abandoned & NA & NA & 11 & 12\\
\hline
grazed/mown & NA & NA & NA & 15\\
\hline
grazed/natural & NA & NA & NA & 64\\
\hline
grazedazed & NA & NA & NA & 1\\
\hline
grazing & NA & NA & NA & 10\\
\hline
irregularly mown & NA & 1 & NA & NA\\
\hline
L & NA & NA & NA & 12\\
\hline
LAS & NA & NA & NA & 12\\
\hline
mown & NA & 361 & 11 & 6640\\
\hline
Mown & NA & 34 & NA & 2\\
\hline
mown and grazed & NA & NA & NA & 178\\
\hline
Mown and grazed & NA & NA & NA & 15\\
\hline
mown/abandoned & NA & NA & NA & 10\\
\hline
mown/grazed & NA & 7 & NA & NA\\
\hline
N & NA & 28 & NA & NA\\
\hline
natural grassland & NA & 1596 & 17 & 104525\\
\hline
Natural grassland & NA & 17 & NA & NA\\
\hline
natural grassland, grazed & NA & NA & NA & 78\\
\hline
P & NA & NA & NA & 48\\
\hline
trampled & NA & 1 & NA & NA\\
\hline
NA & NA & 62 & 3 & 16031\\
\hline
\end{tabular}

\subparagraph{\texorpdfstring{\emph{Suggestions}}{Suggestions}}\label{suggestions-6}

\begin{itemize}
\tightlist
\item
  Reclassify those plots as burn = 1
\end{itemize}

\paragraph{Ex arable (1/0) (column 102 / column
CX)}\label{ex-arable-10-column-102-column-cx}

\subparagraph{\texorpdfstring{\emph{Problems}}{Problems}}\label{problems-7}

\begin{itemize}
\tightlist
\item
  Plots classified as abandonned have ex arable = 0 or ex arable = NA
\end{itemize}

\begin{tabular}{>{\raggedright\arraybackslash}p{8cm}|r|r|r}
\hline
Land use (5 standard categories: mown, grazed, abandoned, natural grassland, NA) & 0 & 1 & <NA>\\
\hline
0 intensive grazing & NA & NA & 1\\
\hline
0t visible & NA & NA & 2\\
\hline
abandoned & 317 & 17 & 15730\\
\hline
Abandoned & NA & NA & 67\\
\hline
abandoned? & NA & NA & 6\\
\hline
abandonment & NA & NA & 1\\
\hline
abondonment & NA & NA & 4\\
\hline
burnt, abandoned & NA & NA & 13\\
\hline
grazed & 3081 & 6 & 30860\\
\hline
grazed and mown & NA & NA & 4\\
\hline
grazed, burnt & NA & NA & 26\\
\hline
grazed, mown & NA & 20 & 120\\
\hline
grazed/abandoned & NA & NA & 23\\
\hline
grazed/mown & NA & NA & 15\\
\hline
grazed/natural & NA & NA & 64\\
\hline
grazedazed & NA & NA & 1\\
\hline
grazing & NA & NA & 10\\
\hline
irregularly mown & 1 & NA & NA\\
\hline
L & NA & NA & 12\\
\hline
LAS & NA & NA & 12\\
\hline
mown & 153 & 30 & 6829\\
\hline
Mown & NA & NA & 36\\
\hline
mown and grazed & NA & NA & 178\\
\hline
Mown and grazed & NA & NA & 15\\
\hline
mown/abandoned & NA & NA & 10\\
\hline
mown/grazed & NA & NA & 7\\
\hline
N & NA & NA & 28\\
\hline
natural grassland & 1354 & 13 & 104771\\
\hline
Natural grassland & NA & NA & 17\\
\hline
natural grassland, grazed & NA & NA & 78\\
\hline
P & NA & NA & 48\\
\hline
trampled & 1 & NA & NA\\
\hline
NA & 37 & 11 & 16048\\
\hline
\end{tabular}

\subparagraph{\texorpdfstring{\emph{Suggestions}}{Suggestions}}\label{suggestions-7}

\begin{itemize}
\tightlist
\item
  Checking the definition of abandonement and ex arable used in the
  database
\end{itemize}

\subparagraph{Ex arable years (column 103 / column CY) and Fertilized
(1/0) (column 104 / column
CZ)}\label{ex-arable-years-column-103-column-cy-and-fertilized-10-column-104-column-cz}

This column should be checked against the column 96 manually to ensure
the validity of the data. Note that there are impossible values for the
Fertilized column (0.3 and 0.5) If the data were provided as is by the
data owner, there is not much we can do (the data owner could have
provided that information without filling the information in column 96)


\end{document}
