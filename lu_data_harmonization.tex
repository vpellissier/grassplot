\documentclass[table]{article}
\usepackage{lmodern}
\usepackage{amssymb,amsmath}
\usepackage{ifxetex,ifluatex}
\usepackage{fixltx2e} % provides \textsubscript
\ifnum 0\ifxetex 1\fi\ifluatex 1\fi=0 % if pdftex
  \usepackage[T1]{fontenc}
  \usepackage[utf8]{inputenc}
\else % if luatex or xelatex
  \ifxetex
    \usepackage{mathspec}
  \else
    \usepackage{fontspec}
  \fi
  \defaultfontfeatures{Ligatures=TeX,Scale=MatchLowercase}
\fi
% use upquote if available, for straight quotes in verbatim environments
\IfFileExists{upquote.sty}{\usepackage{upquote}}{}
% use microtype if available
\IfFileExists{microtype.sty}{%
\usepackage{microtype}
\UseMicrotypeSet[protrusion]{basicmath} % disable protrusion for tt fonts
}{}
\usepackage[margin=1in]{geometry}
\usepackage{hyperref}
\hypersetup{unicode=true,
            pdftitle={Land-use data harmonization},
            pdfauthor={Vincent Pellissier},
            pdfborder={0 0 0},
            breaklinks=true}
\urlstyle{same}  % don't use monospace font for urls
\usepackage{color}
\usepackage{fancyvrb}
\newcommand{\VerbBar}{|}
\newcommand{\VERB}{\Verb[commandchars=\\\{\}]}
\DefineVerbatimEnvironment{Highlighting}{Verbatim}{commandchars=\\\{\}}
% Add ',fontsize=\small' for more characters per line
\usepackage{framed}
\definecolor{shadecolor}{RGB}{248,248,248}
\newenvironment{Shaded}{\begin{snugshade}}{\end{snugshade}}
\newcommand{\KeywordTok}[1]{\textcolor[rgb]{0.13,0.29,0.53}{\textbf{#1}}}
\newcommand{\DataTypeTok}[1]{\textcolor[rgb]{0.13,0.29,0.53}{#1}}
\newcommand{\DecValTok}[1]{\textcolor[rgb]{0.00,0.00,0.81}{#1}}
\newcommand{\BaseNTok}[1]{\textcolor[rgb]{0.00,0.00,0.81}{#1}}
\newcommand{\FloatTok}[1]{\textcolor[rgb]{0.00,0.00,0.81}{#1}}
\newcommand{\ConstantTok}[1]{\textcolor[rgb]{0.00,0.00,0.00}{#1}}
\newcommand{\CharTok}[1]{\textcolor[rgb]{0.31,0.60,0.02}{#1}}
\newcommand{\SpecialCharTok}[1]{\textcolor[rgb]{0.00,0.00,0.00}{#1}}
\newcommand{\StringTok}[1]{\textcolor[rgb]{0.31,0.60,0.02}{#1}}
\newcommand{\VerbatimStringTok}[1]{\textcolor[rgb]{0.31,0.60,0.02}{#1}}
\newcommand{\SpecialStringTok}[1]{\textcolor[rgb]{0.31,0.60,0.02}{#1}}
\newcommand{\ImportTok}[1]{#1}
\newcommand{\CommentTok}[1]{\textcolor[rgb]{0.56,0.35,0.01}{\textit{#1}}}
\newcommand{\DocumentationTok}[1]{\textcolor[rgb]{0.56,0.35,0.01}{\textbf{\textit{#1}}}}
\newcommand{\AnnotationTok}[1]{\textcolor[rgb]{0.56,0.35,0.01}{\textbf{\textit{#1}}}}
\newcommand{\CommentVarTok}[1]{\textcolor[rgb]{0.56,0.35,0.01}{\textbf{\textit{#1}}}}
\newcommand{\OtherTok}[1]{\textcolor[rgb]{0.56,0.35,0.01}{#1}}
\newcommand{\FunctionTok}[1]{\textcolor[rgb]{0.00,0.00,0.00}{#1}}
\newcommand{\VariableTok}[1]{\textcolor[rgb]{0.00,0.00,0.00}{#1}}
\newcommand{\ControlFlowTok}[1]{\textcolor[rgb]{0.13,0.29,0.53}{\textbf{#1}}}
\newcommand{\OperatorTok}[1]{\textcolor[rgb]{0.81,0.36,0.00}{\textbf{#1}}}
\newcommand{\BuiltInTok}[1]{#1}
\newcommand{\ExtensionTok}[1]{#1}
\newcommand{\PreprocessorTok}[1]{\textcolor[rgb]{0.56,0.35,0.01}{\textit{#1}}}
\newcommand{\AttributeTok}[1]{\textcolor[rgb]{0.77,0.63,0.00}{#1}}
\newcommand{\RegionMarkerTok}[1]{#1}
\newcommand{\InformationTok}[1]{\textcolor[rgb]{0.56,0.35,0.01}{\textbf{\textit{#1}}}}
\newcommand{\WarningTok}[1]{\textcolor[rgb]{0.56,0.35,0.01}{\textbf{\textit{#1}}}}
\newcommand{\AlertTok}[1]{\textcolor[rgb]{0.94,0.16,0.16}{#1}}
\newcommand{\ErrorTok}[1]{\textcolor[rgb]{0.64,0.00,0.00}{\textbf{#1}}}
\newcommand{\NormalTok}[1]{#1}
\usepackage{graphicx,grffile}
\makeatletter
\def\maxwidth{\ifdim\Gin@nat@width>\linewidth\linewidth\else\Gin@nat@width\fi}
\def\maxheight{\ifdim\Gin@nat@height>\textheight\textheight\else\Gin@nat@height\fi}
\makeatother
% Scale images if necessary, so that they will not overflow the page
% margins by default, and it is still possible to overwrite the defaults
% using explicit options in \includegraphics[width, height, ...]{}
\setkeys{Gin}{width=\maxwidth,height=\maxheight,keepaspectratio}
\IfFileExists{parskip.sty}{%
\usepackage{parskip}
}{% else
\setlength{\parindent}{0pt}
\setlength{\parskip}{6pt plus 2pt minus 1pt}
}
\setlength{\emergencystretch}{3em}  % prevent overfull lines
\providecommand{\tightlist}{%
  \setlength{\itemsep}{0pt}\setlength{\parskip}{0pt}}
\setcounter{secnumdepth}{0}
% Redefines (sub)paragraphs to behave more like sections
\ifx\paragraph\undefined\else
\let\oldparagraph\paragraph
\renewcommand{\paragraph}[1]{\oldparagraph{#1}\mbox{}}
\fi
\ifx\subparagraph\undefined\else
\let\oldsubparagraph\subparagraph
\renewcommand{\subparagraph}[1]{\oldsubparagraph{#1}\mbox{}}
\fi

%%% Use protect on footnotes to avoid problems with footnotes in titles
\let\rmarkdownfootnote\footnote%
\def\footnote{\protect\rmarkdownfootnote}

%%% Change title format to be more compact
\usepackage{titling}

% Create subtitle command for use in maketitle
\newcommand{\subtitle}[1]{
  \posttitle{
    \begin{center}\large#1\end{center}
    }
}

\setlength{\droptitle}{-2em}
  \title{Land-use data harmonization}
  \pretitle{\vspace{\droptitle}\centering\huge}
  \posttitle{\par}
  \author{Vincent Pellissier}
  \preauthor{\centering\large\emph}
  \postauthor{\par}
  \predate{\centering\large\emph}
  \postdate{\par}
  \date{8 November 2018}


\begin{document}
\maketitle

This report explains the steps taken to correct the land-use data. The
masterfile used is the version 1.8, as found on Novembre 8th 2018. The
sheet in Grassplot 1.8.xlsx containing the LU information is the sheet
`datasets' and was read and saved as a .rds file to ease the process
(faster loading)

\begin{Shaded}
\begin{Highlighting}[]
\NormalTok{df <-}\StringTok{ }\KeywordTok{readRDS}\NormalTok{(}\KeywordTok{file.path}\NormalTok{(path_grassplot, }\StringTok{'Grassplot 1.8_Data.rds'}\NormalTok{))}
\NormalTok{df <-}\StringTok{ }\NormalTok{df }\OperatorTok
\StringTok{        }\KeywordTok{mutate_at}\NormalTok{(}\DataTypeTok{.vars =} \KeywordTok{c}\NormalTok{(}\DecValTok{95}\OperatorTok{:}\DecValTok{104}\NormalTok{), }\KeywordTok{funs}\NormalTok{(}\KeywordTok{ifelse}\NormalTok{(. }\OperatorTok\StringTok{ }\KeywordTok{c}\NormalTok{(}\StringTok{'NA'}\NormalTok{, }\StringTok{'[NA]'}\NormalTok{), }\OtherTok{NA}\NormalTok{, .)))}
\end{Highlighting}
\end{Shaded}

\subsubsection{Correction of the land use
column}\label{correction-of-the-land-use-column}

The file lookup\_table\_LU.xlsx contains two sheets: *
global\_land\_use: lookuptable matching the values in column 95 with a
new classification. This new classification contains the 5 original
values (mown, grazed, abandoned, natural grassland, NA), or combinations
of thereof (separated by `/') in case of a mixed land-use. This
classification is to be discussed, but as it is can still be separated
in binary variables if necessary. * land\_use\_detail: lookup table
between the column 96 and other column. This needs to be discussed
further before I finalize the table.

\begin{Shaded}
\begin{Highlighting}[]
\NormalTok{lut <-}\StringTok{ }\KeywordTok{read_excel}\NormalTok{(}\KeywordTok{file.path}\NormalTok{(path_grassplot, }\StringTok{"lookup_table_LU.xlsx"}\NormalTok{ ),}
                 \DataTypeTok{sheet =} \StringTok{'global_land_use'}\NormalTok{)}
\end{Highlighting}
\end{Shaded}

The global\_land\_use lookup table is used to harmonize the column 95.
Note that a new column (land\_use) is created to prevent unwanted
information removal:

\begin{Shaded}
\begin{Highlighting}[]
\NormalTok{df <-}\StringTok{ }\NormalTok{df }\OperatorTok
\StringTok{    }\KeywordTok{left_join}\NormalTok{(lut, }\DataTypeTok{by =} \KeywordTok{c}\NormalTok{(}\StringTok{"Land use (5 standard categories: mown, grazed, abandoned, natural grassland, NA)"}\NormalTok{ =}\StringTok{ "old_classification"}\NormalTok{))}
\end{Highlighting}
\end{Shaded}

\subsubsection{Identification of problematic
datasets}\label{identification-of-problematic-datasets}

In this step, we identify datasets with discrepancies to * Manually
check in the original datasets * Correct the discrepancies

\paragraph{Mowing and mowing
intensity}\label{mowing-and-mowing-intensity}

Datasets containing plots for which the land\_use is mowing (alone or in
combination) and grazing or mowing intensity is 0 or NA

\begin{Shaded}
\begin{Highlighting}[]
\NormalTok{df }\OperatorTok\StringTok{ }
\StringTok{    }\KeywordTok{filter}\NormalTok{(}\KeywordTok{grepl}\NormalTok{(}\StringTok{"mown"}\NormalTok{, land_use)) }\OperatorTok
\StringTok{    }\KeywordTok{filter}\NormalTok{(}\StringTok{`}\DataTypeTok{Mowing (1/0)}\StringTok{`} \OperatorTok{!=}\StringTok{ '1'} \OperatorTok{|}\StringTok{ }
\StringTok{               }\KeywordTok{is.na}\NormalTok{(}\StringTok{`}\DataTypeTok{Mowing (1/0)}\StringTok{`}\NormalTok{) }\OperatorTok{|}\StringTok{ }
\StringTok{                         `}\DataTypeTok{Mowing frequency: cuts per year (2=2cut/yr; 1=1cut/yr; 0.5=1 cut/2 yr or 2 yr abandoned; 0.2=1 cut/5 yr or 5 yr abandones; 0=never mown)}\StringTok{`} \OperatorTok{==}\StringTok{ '0'} \OperatorTok{|}
\StringTok{               }\KeywordTok{is.na}\NormalTok{(}\StringTok{`}\DataTypeTok{Mowing frequency: cuts per year (2=2cut/yr; 1=1cut/yr; 0.5=1 cut/2 yr or 2 yr abandoned; 0.2=1 cut/5 yr or 5 yr abandones; 0=never mown)}\StringTok{`}\NormalTok{))}\OperatorTok
\StringTok{    }\KeywordTok{distinct}\NormalTok{(}\StringTok{`}\DataTypeTok{Dataset ID}\StringTok{`}\NormalTok{)}\OperatorTok
\StringTok{    }\KeywordTok{pull}\NormalTok{()}
\end{Highlighting}
\end{Shaded}

\begin{verbatim}
##  [1] "AS_A" "AT_A" "BG_A" "CH_A" "CH_B" "CH_C" "CZ_A" "CZ_B" "CZ_C" "CZ_H"
## [11] "CZ_I" "DE_A" "DE_E" "DE_F" "DE_H" "DE_I" "DE_J" "DE_L" "DE_N" "DE_O"
## [21] "DE_P" "ES_A" "ES_C" "EU_A" "EU_B" "EU_C" "EU_E" "EU_G" "EU_J" "FR_A"
## [31] "HR_A" "HU_A" "HU_C" "IR_A" "IT_H" "IT_K" "JP_A" "LV_A" "NL_A" "PL_A"
## [41] "PL_B" "PL_C" "PL_D" "RO_A" "RO_B" "RS_A" "RU_A" "TJ_A" "UA_C" "UA_D"
## [51] "EU_K"
\end{verbatim}


\end{document}
