\documentclass[table]{article}
\usepackage{lmodern}
\usepackage{amssymb,amsmath}
\usepackage{ifxetex,ifluatex}
\usepackage{fixltx2e} % provides \textsubscript
\ifnum 0\ifxetex 1\fi\ifluatex 1\fi=0 % if pdftex
  \usepackage[T1]{fontenc}
  \usepackage[utf8]{inputenc}
\else % if luatex or xelatex
  \ifxetex
    \usepackage{mathspec}
  \else
    \usepackage{fontspec}
  \fi
  \defaultfontfeatures{Ligatures=TeX,Scale=MatchLowercase}
\fi
% use upquote if available, for straight quotes in verbatim environments
\IfFileExists{upquote.sty}{\usepackage{upquote}}{}
% use microtype if available
\IfFileExists{microtype.sty}{%
\usepackage{microtype}
\UseMicrotypeSet[protrusion]{basicmath} % disable protrusion for tt fonts
}{}
\usepackage[margin=1in]{geometry}
\usepackage{hyperref}
\hypersetup{unicode=true,
            pdftitle={Land-use data harmonization},
            pdfauthor={Vincent Pellissier},
            pdfborder={0 0 0},
            breaklinks=true}
\urlstyle{same}  % don't use monospace font for urls
\usepackage{color}
\usepackage{fancyvrb}
\newcommand{\VerbBar}{|}
\newcommand{\VERB}{\Verb[commandchars=\\\{\}]}
\DefineVerbatimEnvironment{Highlighting}{Verbatim}{commandchars=\\\{\}}
% Add ',fontsize=\small' for more characters per line
\usepackage{framed}
\definecolor{shadecolor}{RGB}{248,248,248}
\newenvironment{Shaded}{\begin{snugshade}}{\end{snugshade}}
\newcommand{\KeywordTok}[1]{\textcolor[rgb]{0.13,0.29,0.53}{\textbf{#1}}}
\newcommand{\DataTypeTok}[1]{\textcolor[rgb]{0.13,0.29,0.53}{#1}}
\newcommand{\DecValTok}[1]{\textcolor[rgb]{0.00,0.00,0.81}{#1}}
\newcommand{\BaseNTok}[1]{\textcolor[rgb]{0.00,0.00,0.81}{#1}}
\newcommand{\FloatTok}[1]{\textcolor[rgb]{0.00,0.00,0.81}{#1}}
\newcommand{\ConstantTok}[1]{\textcolor[rgb]{0.00,0.00,0.00}{#1}}
\newcommand{\CharTok}[1]{\textcolor[rgb]{0.31,0.60,0.02}{#1}}
\newcommand{\SpecialCharTok}[1]{\textcolor[rgb]{0.00,0.00,0.00}{#1}}
\newcommand{\StringTok}[1]{\textcolor[rgb]{0.31,0.60,0.02}{#1}}
\newcommand{\VerbatimStringTok}[1]{\textcolor[rgb]{0.31,0.60,0.02}{#1}}
\newcommand{\SpecialStringTok}[1]{\textcolor[rgb]{0.31,0.60,0.02}{#1}}
\newcommand{\ImportTok}[1]{#1}
\newcommand{\CommentTok}[1]{\textcolor[rgb]{0.56,0.35,0.01}{\textit{#1}}}
\newcommand{\DocumentationTok}[1]{\textcolor[rgb]{0.56,0.35,0.01}{\textbf{\textit{#1}}}}
\newcommand{\AnnotationTok}[1]{\textcolor[rgb]{0.56,0.35,0.01}{\textbf{\textit{#1}}}}
\newcommand{\CommentVarTok}[1]{\textcolor[rgb]{0.56,0.35,0.01}{\textbf{\textit{#1}}}}
\newcommand{\OtherTok}[1]{\textcolor[rgb]{0.56,0.35,0.01}{#1}}
\newcommand{\FunctionTok}[1]{\textcolor[rgb]{0.00,0.00,0.00}{#1}}
\newcommand{\VariableTok}[1]{\textcolor[rgb]{0.00,0.00,0.00}{#1}}
\newcommand{\ControlFlowTok}[1]{\textcolor[rgb]{0.13,0.29,0.53}{\textbf{#1}}}
\newcommand{\OperatorTok}[1]{\textcolor[rgb]{0.81,0.36,0.00}{\textbf{#1}}}
\newcommand{\BuiltInTok}[1]{#1}
\newcommand{\ExtensionTok}[1]{#1}
\newcommand{\PreprocessorTok}[1]{\textcolor[rgb]{0.56,0.35,0.01}{\textit{#1}}}
\newcommand{\AttributeTok}[1]{\textcolor[rgb]{0.77,0.63,0.00}{#1}}
\newcommand{\RegionMarkerTok}[1]{#1}
\newcommand{\InformationTok}[1]{\textcolor[rgb]{0.56,0.35,0.01}{\textbf{\textit{#1}}}}
\newcommand{\WarningTok}[1]{\textcolor[rgb]{0.56,0.35,0.01}{\textbf{\textit{#1}}}}
\newcommand{\AlertTok}[1]{\textcolor[rgb]{0.94,0.16,0.16}{#1}}
\newcommand{\ErrorTok}[1]{\textcolor[rgb]{0.64,0.00,0.00}{\textbf{#1}}}
\newcommand{\NormalTok}[1]{#1}
\usepackage{longtable,booktabs}
\usepackage{graphicx,grffile}
\makeatletter
\def\maxwidth{\ifdim\Gin@nat@width>\linewidth\linewidth\else\Gin@nat@width\fi}
\def\maxheight{\ifdim\Gin@nat@height>\textheight\textheight\else\Gin@nat@height\fi}
\makeatother
% Scale images if necessary, so that they will not overflow the page
% margins by default, and it is still possible to overwrite the defaults
% using explicit options in \includegraphics[width, height, ...]{}
\setkeys{Gin}{width=\maxwidth,height=\maxheight,keepaspectratio}
\IfFileExists{parskip.sty}{%
\usepackage{parskip}
}{% else
\setlength{\parindent}{0pt}
\setlength{\parskip}{6pt plus 2pt minus 1pt}
}
\setlength{\emergencystretch}{3em}  % prevent overfull lines
\providecommand{\tightlist}{%
  \setlength{\itemsep}{0pt}\setlength{\parskip}{0pt}}
\setcounter{secnumdepth}{5}
% Redefines (sub)paragraphs to behave more like sections
\ifx\paragraph\undefined\else
\let\oldparagraph\paragraph
\renewcommand{\paragraph}[1]{\oldparagraph{#1}\mbox{}}
\fi
\ifx\subparagraph\undefined\else
\let\oldsubparagraph\subparagraph
\renewcommand{\subparagraph}[1]{\oldsubparagraph{#1}\mbox{}}
\fi

%%% Use protect on footnotes to avoid problems with footnotes in titles
\let\rmarkdownfootnote\footnote%
\def\footnote{\protect\rmarkdownfootnote}

%%% Change title format to be more compact
\usepackage{titling}

% Create subtitle command for use in maketitle
\newcommand{\subtitle}[1]{
  \posttitle{
    \begin{center}\large#1\end{center}
    }
}

\setlength{\droptitle}{-2em}
  \title{Land-use data harmonization}
  \pretitle{\vspace{\droptitle}\centering\huge}
  \posttitle{\par}
  \author{Vincent Pellissier}
  \preauthor{\centering\large\emph}
  \postauthor{\par}
  \predate{\centering\large\emph}
  \postdate{\par}
  \date{27 November 2018}

\usepackage{float}

\begin{document}
\maketitle

This report explains the steps taken to correct the land-use data. The
masterfile used is the version 1.20, as found on December 4th 2018. The
sheet in Grassplot 1.20.xlsx containing the LU information is the sheet
`datasets' and was read and saved as a .rds file to ease the process
(faster loading)

\begin{Shaded}
\begin{Highlighting}[]
\NormalTok{df <-}\StringTok{ }\KeywordTok{readRDS}\NormalTok{(}\KeywordTok{file.path}\NormalTok{(path_grassplot, }\StringTok{'Grassplot 1.20_Data.rds'}\NormalTok{))}
\NormalTok{df <-}\StringTok{ }\NormalTok{df }\OperatorTok
\StringTok{        }\KeywordTok{mutate_at}\NormalTok{(}\DataTypeTok{.vars =} \KeywordTok{c}\NormalTok{(}\DecValTok{95}\OperatorTok{:}\DecValTok{104}\NormalTok{), }\KeywordTok{funs}\NormalTok{(}\KeywordTok{ifelse}\NormalTok{(. }\OperatorTok\StringTok{ }\KeywordTok{c}\NormalTok{(}\StringTok{'NA'}\NormalTok{, }\StringTok{'[NA]'}\NormalTok{), }\OtherTok{NA}\NormalTok{, .)))}
\end{Highlighting}
\end{Shaded}

The columns 98 (mowing frequency) and 100 (grazing intensity) are
renamed as \texttt{mowing\_frequency} and \texttt{grazing\ intensity}:

\begin{Shaded}
\begin{Highlighting}[]
\KeywordTok{names}\NormalTok{(df)[}\KeywordTok{which}\NormalTok{(}\KeywordTok{names}\NormalTok{(df) }\OperatorTok\StringTok{ }\KeywordTok{c}\NormalTok{(}\StringTok{'Mowing frequency: cuts per year (2=2cut/yr; 1=1cut/yr; 0.5=1 cut/2 yr or 2 yr abandoned; 0.2=1 cut/5 yr or 5 yr abandones; 0=never mown)'}\NormalTok{, }\StringTok{'Grazing intensity (0/1) 1= intensive grazing'}\NormalTok{))] <-}\StringTok{ }\KeywordTok{c}\NormalTok{(}\StringTok{'mowing_frequency'}\NormalTok{, }\StringTok{'grazing_intensity'}\NormalTok{)}
\end{Highlighting}
\end{Shaded}

The column 104 (Fertilized (1/0)) should be a binary column. Yet, it
contains the values c(``0'', ``1'', ``0.5'', ``0.3''), meaning that it
is an intensity column. The column is thus renamed
\texttt{fertilization\_intensity}

\begin{Shaded}
\begin{Highlighting}[]
\KeywordTok{names}\NormalTok{(df)[}\KeywordTok{which}\NormalTok{(}\KeywordTok{names}\NormalTok{(df) }\OperatorTok{==}\StringTok{ 'Fertilized (1/0)'}\NormalTok{)] <-}\StringTok{ 'fertilization_intensity'}
\end{Highlighting}
\end{Shaded}

The column 103 (Ex arable years (years since last ploughing)) is renamed
\texttt{years\_abandonment}

\begin{Shaded}
\begin{Highlighting}[]
\KeywordTok{names}\NormalTok{(df)[}\KeywordTok{which}\NormalTok{(}\KeywordTok{names}\NormalTok{(df) }\OperatorTok{==}\StringTok{ 'Ex arable years (years since last ploughing)'}\NormalTok{)] <-}\StringTok{ 'years_abandonment'}
\end{Highlighting}
\end{Shaded}

\textbf{PRELIMINARY NOTE: there are duplicated ID in the column
\texttt{Grassplot\ ID\ of\ plot}. The following ID are duplicated
DE\_B\_N001\_0.0001aa, NA}

\section{Correction of the land use
column}\label{correction-of-the-land-use-column}

The file lookup\_table\_LU.xlsx contains two sheets:

\begin{itemize}
\tightlist
\item
  land\_use\_detail: lookup table between the column 96 and other
  column. This needs to be discussed further before I finalize the
  table.
\item
  global\_land\_use: lookuptable matching the values in column 95 with a
  new classification. This new classification is a set of binary columns
  (allowing for mixed land-use) plus one declarative column. Each binary
  column correspond to a management practice (grazed, mown, burnt,
  fertilized, abandoned, natural\_grassland), the declarative column
  stores additional information that cannot be put in one of the 6
  columns (\emph{i.e} trampled) and that will be used afterward.
  \textbf{NOTE} LAS, P, L were noted as land-use for the dataset PL\_C
  and N was noted as land-use for the dataset RU\_J. I could not find
  what these stand for. I assumed that these plots don't have a land-use
  but the info is still in the column `other'.
\end{itemize}

\begin{Shaded}
\begin{Highlighting}[]
\NormalTok{lut <-}\StringTok{ }\KeywordTok{read_excel}\NormalTok{(}\KeywordTok{file.path}\NormalTok{(path_grassplot, }\StringTok{"lookup_table_LU.xlsx"}\NormalTok{ ),}
                 \DataTypeTok{sheet =} \StringTok{'global_land_use'}\NormalTok{)}
\end{Highlighting}
\end{Shaded}

The global\_land\_use lookup table is used to harmonize the column 95
with the new binary columns:

\begin{Shaded}
\begin{Highlighting}[]
\NormalTok{df <-}\StringTok{ }\NormalTok{df }\OperatorTok
\StringTok{    }\KeywordTok{left_join}\NormalTok{(lut, }
              \DataTypeTok{by =} \KeywordTok{c}\NormalTok{(}\StringTok{"Land use (5 standard categories: mown, grazed, abandoned, natural grassland, NA)"}\NormalTok{ =}\StringTok{ "old_classification"}\NormalTok{))}
\end{Highlighting}
\end{Shaded}

\section{Correction of the intensity/frequency/duration
columns}\label{correction-of-the-intensityfrequencyduration-columns}

These three columns (mowing\_frequency, grazing\_intensity,
fertilization\_intensity) are presently coded as text columns\\
The column mowing\_frequency contains 2 non-numeric values :
x,fluctuating mowing and abandonment decades before the sampling.\\
The column grazing\_intensity contains 4 non-numeric values : low,
middle, high, overgrazing.\\
The column fertilization\_intensity contains 0 non-numeric values .\\
The column years\_abandonment contains 0 non-numeric values .

\subsection{Mowing frequency}\label{mowing-frequency}

The mowing\_frequency does have non-numerical values.\\
A closer look (manual) at the master files shows that there is some
information in the column 96, that can be converted into
mowing\_frequency.\\
The column is converted to numerical, meaning that we lose the
information for now, but it will be retrieved later using the column 96.

\begin{Shaded}
\begin{Highlighting}[]
\NormalTok{df}\OperatorTok{$}\NormalTok{mowing_frequency <-}\StringTok{ }\KeywordTok{as.numeric}\NormalTok{(df}\OperatorTok{$}\NormalTok{mowing_frequency)}
\end{Highlighting}
\end{Shaded}

\textbf{Post correction check}: original number of plots with frequency
- present number of plots with frequency (should be 0) = 703.
\textbf{Manual note} The difference here is due to the removal of the
plots with mowing\_frequency ==`x'.

\subsection{Grazing intensity}\label{grazing-intensity}

The grazing\_intensity does have non-numerical values. The verbal
descriptors low, middle, high, overgrazing are replaced by numerical
values and the column is converted to numerical:

\begin{itemize}
\tightlist
\item
  \texttt{overgrazing} and \texttt{high} are replaced by 1
\item
  \texttt{middle} is replaced by 0.5
\item
  \texttt{low} is replaced by 0.1
\end{itemize}

\begin{Shaded}
\begin{Highlighting}[]
\NormalTok{df[df}\OperatorTok{$}\NormalTok{grazing_intensity }\OperatorTok\StringTok{ }\KeywordTok{c}\NormalTok{(}\StringTok{'overgrazing'}\NormalTok{, }\StringTok{'high'}\NormalTok{),]}\OperatorTok
\StringTok{  }\KeywordTok{mutate}\NormalTok{(}\DataTypeTok{grazing_intensity =} \DecValTok{1}\NormalTok{)}

\NormalTok{df[df}\OperatorTok{$}\NormalTok{grazing_intensity }\OperatorTok\StringTok{ }\KeywordTok{c}\NormalTok{(}\StringTok{'middle'}\NormalTok{),]}\OperatorTok
\StringTok{  }\KeywordTok{mutate}\NormalTok{(}\DataTypeTok{grazing_intensity =} \FloatTok{0.5}\NormalTok{)}

\NormalTok{df[df}\OperatorTok{$}\NormalTok{grazing_intensity }\OperatorTok\StringTok{ }\KeywordTok{c}\NormalTok{(}\StringTok{'low'}\NormalTok{),]}\OperatorTok
\StringTok{  }\KeywordTok{mutate}\NormalTok{(}\DataTypeTok{grazing_intensity =} \FloatTok{0.1}\NormalTok{)}

\NormalTok{df}\OperatorTok{$}\NormalTok{grazing_intensity <-}\StringTok{ }\KeywordTok{as.numeric}\NormalTok{(df}\OperatorTok{$}\NormalTok{grazing_intensity)}
\end{Highlighting}
\end{Shaded}

\textbf{Post correction check}: original number of plots with intensity
- present number of plots with intensity (should be 0) = 0.

\subsection{Fertilization intensity}\label{fertilization-intensity}

The fertilization\_intensity column does not have non-numerical
values.\\
The column is converted to numerical.

\begin{Shaded}
\begin{Highlighting}[]
\NormalTok{df}\OperatorTok{$}\NormalTok{fertilization_intensity <-}\StringTok{ }\KeywordTok{as.numeric}\NormalTok{(df}\OperatorTok{$}\NormalTok{fertilization_intensity)}
\end{Highlighting}
\end{Shaded}

\textbf{Post correction check}: original number of plots with intensity
- present number of plots with intensity (should be 0) = -334.

\subsection{Abandonment length}\label{abandonment-length}

The year\_abandonment column does not have non-numerical values. The
column is converted to numeric.

\begin{Shaded}
\begin{Highlighting}[]
\NormalTok{df}\OperatorTok{$}\NormalTok{years_abandonment <-}\StringTok{ }\KeywordTok{as.numeric}\NormalTok{(df}\OperatorTok{$}\NormalTok{years_abandonment)}
\end{Highlighting}
\end{Shaded}

\section{Matching new binary columns and frequency/intensity
columns.}\label{matching-new-binary-columns-and-frequencyintensity-columns.}

In this step, we identify datasets with discrepancies between the new
binary columns and the matching intensity column to:

\begin{itemize}
\tightlist
\item
  Manually check in the original datasets or publications
\item
  Correct the discrepancies.
\end{itemize}

\textbf{NOTE} in the following, I assumed that if the
intensity/frequency/duration column was non empty, but the binary was,
the intensity/frequency/duration column takes precedence (\emph{i.e} if
the frequency/intensity/duration, the binary column \textbf{HAS} to be
== 1)

\subsection{Mowing and mowing
intensity}\label{mowing-and-mowing-intensity}

\subsubsection{\texorpdfstring{\emph{Datasets containing plots for which
mowing intensity \textgreater{} 0 \& mown != 1 (here and after, we refer
to the new binary
columns)}}{Datasets containing plots for which mowing intensity \textgreater{} 0 \& mown != 1 (here and after, we refer to the new binary columns)}}\label{datasets-containing-plots-for-which-mowing-intensity-0-mown-1-here-and-after-we-refer-to-the-new-binary-columns}

\begin{itemize}
\tightlist
\item
  The datasets CZ\_J, DE\_S, EU\_K contain 237 plots having a mowing
  intensity but classified as mown == 0. These plots have been manually
  checked the 8/11/2018 (masterfile v1.8) and contains not errors or
  additional information. They are thus reclassified as mown == 1.\\
  \textbf{NOTE} If necessary, we can discuss a frequency threshold the
  classify a plot as mown or not.
\end{itemize}

\begin{Shaded}
\begin{Highlighting}[]
\NormalTok{df[df}\OperatorTok{$}\NormalTok{mowing_frequency }\OperatorTok{!=}\StringTok{ }\DecValTok{0} \OperatorTok{&}\StringTok{ }\OperatorTok{!}\KeywordTok{is.na}\NormalTok{(df}\OperatorTok{$}\NormalTok{mowing_frequency) }\OperatorTok{&}\StringTok{ }\NormalTok{df}\OperatorTok{$}\NormalTok{mown }\OperatorTok{==}\StringTok{ }\DecValTok{0}\NormalTok{,] }\OperatorTok\StringTok{ }
\StringTok{  }\KeywordTok{mutate}\NormalTok{(}\DataTypeTok{mown =} \DecValTok{1}\NormalTok{)}
\end{Highlighting}
\end{Shaded}

\textbf{Post reclassification check:} number of plots with a discrepancy
- number of plots reclassified (should be 0) = 0

\subsection{Grazing and Grazing
intensity}\label{grazing-and-grazing-intensity}

\subsubsection{\texorpdfstring{\emph{Datasets containing plots for which
grazing intensity \textgreater{} 0 \& grazed !=
1:}}{Datasets containing plots for which grazing intensity \textgreater{} 0 \& grazed != 1:}}\label{datasets-containing-plots-for-which-grazing-intensity-0-grazed-1}

\begin{itemize}
\tightlist
\item
  The datasets AZ\_B, EU\_K, IR\_A, PL\_D, TR\_B, UA\_G contain 1189
  plots having a mowing intensity but classified as grazed == 0. These
  plots have been manually checked the 8/11/2018 (masterfile v1.8) and
  contains not errors or additional information. They are thus
  reclassified as grazed == 1.
\end{itemize}

\begin{Shaded}
\begin{Highlighting}[]
\NormalTok{df[df}\OperatorTok{$}\NormalTok{grazing_intensity }\OperatorTok{!=}\StringTok{ }\DecValTok{0} \OperatorTok{&}\StringTok{ }\OperatorTok{!}\KeywordTok{is.na}\NormalTok{(df}\OperatorTok{$}\NormalTok{grazing_intensity) }\OperatorTok{&}\StringTok{ }\NormalTok{df}\OperatorTok{$}\NormalTok{grazed }\OperatorTok{==}\StringTok{ }\DecValTok{0}\NormalTok{,] }\OperatorTok\StringTok{ }
\StringTok{  }\KeywordTok{mutate}\NormalTok{(}\DataTypeTok{grazed =} \DecValTok{1}\NormalTok{)}
\end{Highlighting}
\end{Shaded}

\textbf{Post reclassification check:} number of plots with a discrepancy
- number of plots reclassified (should be 0) = 0

\subsection{Fertilization and fertiliation
intensity}\label{fertilization-and-fertiliation-intensity}

\subsubsection{\texorpdfstring{\emph{Datasets containing plots for which
fertilization intensity \textgreater{} 0 \& fertilized != 1 (here and
after, we refer to the new binary
columns)}}{Datasets containing plots for which fertilization intensity \textgreater{} 0 \& fertilized != 1 (here and after, we refer to the new binary columns)}}\label{datasets-containing-plots-for-which-fertilization-intensity-0-fertilized-1-here-and-after-we-refer-to-the-new-binary-columns}

\begin{itemize}
\tightlist
\item
  The datasets DE\_S, ES\_P, EU\_K, PL\_D, TR\_A, TR\_B contain 433
  plots having a fertilization intensity but classified as fertilized ==
  0. These plots have \textbf{not} been manually verified (to be done).
  They are thus reclassified as fertilized == 1.
\end{itemize}

\begin{Shaded}
\begin{Highlighting}[]
\NormalTok{df[df}\OperatorTok{$}\NormalTok{fertilization_intensity }\OperatorTok{!=}\StringTok{ }\DecValTok{0} \OperatorTok{&}\StringTok{ }\OperatorTok{!}\KeywordTok{is.na}\NormalTok{(df}\OperatorTok{$}\NormalTok{fertilization_intensity) }\OperatorTok{&}\StringTok{ }\NormalTok{df}\OperatorTok{$}\NormalTok{fertilized }\OperatorTok{==}\StringTok{ }\DecValTok{0}\NormalTok{,] }\OperatorTok\StringTok{ }
\StringTok{  }\KeywordTok{mutate}\NormalTok{(}\DataTypeTok{fertilized =} \DecValTok{1}\NormalTok{)}
\end{Highlighting}
\end{Shaded}

\textbf{Post reclassification check:} number of plots with a discrepancy
- number of plots reclassified (should be 0) = 0

\section{Matching new and old binary
columns}\label{matching-new-and-old-binary-columns}

In this step, we identify and correct discrepancies between the newly
created and corrected binary column, and the former ones. Here, we
consider that mown == 1 or grazed == 1 is always correct, since it is
based on the above correction (based either on the broad land-use or on
the mowing / grazing intensity). Thus, we only identify (and potentialy
correct) plots for which the old binary column is 0 or NA and the new
one is not 0 or NA.

\subsection{Mowing}\label{mowing}

\begin{table}[H]

\caption{\label{tab:unnamed-chunk-18}Contingency table of columns mown and Mowing (1/0)}
\centering
\begin{tabular}[t]{l|r|r|r}
\hline
Mowing (1/0) & 0 & 1 & <NA>\\
\hline
? & NA & NA & 6\\
\hline
0 & 7990 & 213 & 22\\
\hline
1 & NA & 778 & NA\\
\hline
abandoned & NA & NA & 242\\
\hline
lawn mown 5-10/yr & NA & NA & 78\\
\hline
meadow abandoned c. 5 yr ago & NA & NA & 78\\
\hline
meadow mown 1/yr & NA & NA & 78\\
\hline
mown & NA & NA & 446\\
\hline
x & NA & NA & 18\\
\hline
NA & 149724 & 6067 & 22233\\
\hline
\end{tabular}
\end{table}

\subsubsection{Datasets with discrepancies between mown and Mowing (1/0)
(automatic
report):}\label{datasets-with-discrepancies-between-mown-and-mowing-10-automatic-report}

The dataset(s) AT\_E contain(s) 18 plots with mown = NA and Mowing (1/0)
= x

The dataset(s) DE\_F contain(s) 78 plots with mown = NA and Mowing (1/0)
= meadow mown 1/yr

The dataset(s) DE\_F contain(s) 78 plots with mown = NA and Mowing (1/0)
= lawn mown 5-10/yr

The dataset(s) DE\_F contain(s) 78 plots with mown = NA and Mowing (1/0)
= meadow abandoned c. 5 yr ago

The dataset(s) EU\_C contain(s) 446 plots with mown = NA and Mowing
(1/0) = mown

The dataset(s) EU\_C contain(s) 242 plots with mown = NA and Mowing
(1/0) = abandoned

The dataset(s) PL\_A contain(s) 6 plots with mown = NA and Mowing (1/0)
= ?

\paragraph{Manual verification of discrepancies (has to be updated
manually if necessary for each version of the
masterfile)}\label{manual-verification-of-discrepancies-has-to-be-updated-manually-if-necessary-for-each-version-of-the-masterfile}

\subparagraph{\texorpdfstring{\emph{Datasets containing plots with mown
= NA \& Mowing (1/0) == ? (new vs former
column):}}{Datasets containing plots with mown = NA \& Mowing (1/0) == ? (new vs former column):}}\label{datasets-containing-plots-with-mown-na-mowing-10-new-vs-former-column}

The following plots in PL\_A have mown = NA and Mowing (1/0) == `?'

\begin{verbatim}
## [1] "PL_A_N003_0.0001a" "PL_A_N003_0.001a"  "PL_A_N003_0.01a"  
## [4] "PL_A_N003_0.1a"    "PL_A_N003_10a"     "PL_A_N003_1a"
\end{verbatim}

No additional information could be found, so the plots are left with
mown = NA

\paragraph{Check}\label{check}

\begin{table}[H]

\caption{\label{tab:unnamed-chunk-21}Contingency table of columns mown and Mowing (1/0) after reclassification (check)}
\centering
\begin{tabular}[t]{l|r|r|r}
\hline
Mowing (1/0) & 0 & 1 & <NA>\\
\hline
? & NA & NA & 6\\
\hline
0 & 7990 & 213 & 22\\
\hline
1 & NA & 778 & NA\\
\hline
abandoned & NA & NA & 242\\
\hline
lawn mown 5-10/yr & NA & NA & 78\\
\hline
meadow abandoned c. 5 yr ago & NA & NA & 78\\
\hline
meadow mown 1/yr & NA & NA & 78\\
\hline
mown & NA & NA & 446\\
\hline
x & NA & NA & 18\\
\hline
NA & 149724 & 6067 & 22233\\
\hline
\end{tabular}
\end{table}

\subsection{Grazing}\label{grazing}

\begin{table}[H]

\caption{\label{tab:unnamed-chunk-22}Contingency table of columns grazed and Grazing (1/0)}
\centering
\begin{tabular}[t]{l|r|r|r}
\hline
Grazing (1/0) & 0 & 1 & <NA>\\
\hline
? & NA & 7 & 6\\
\hline
0 & 3585 & 13 & 17\\
\hline
1 & 136 & 6813 & NA\\
\hline
2 & NA & 1 & NA\\
\hline
probably & 6 & NA & NA\\
\hline
NA & 124921 & 29290 & 23178\\
\hline
\end{tabular}
\end{table}

\subsubsection{Datasets with discrepancies between grazed and Grazing
(1/0) (automatic
report):}\label{datasets-with-discrepancies-between-grazed-and-grazing-10-automatic-report}

The dataset(s) PL\_A contain(s) 6 plots with grazed = NA and Grazing
(1/0) = ?

The dataset(s) PL\_A contain(s) 6 plots with grazed == 0 and Grazing
(1/0) = probably

The dataset(s) PL\_A, UA\_L contain(s) 136 plots with grazed == 0 and
Grazing (1/0) = 1

\paragraph{Manual verification of discrepancies (has to be updated
manually if necessary for each version of the
masterfile)}\label{manual-verification-of-discrepancies-has-to-be-updated-manually-if-necessary-for-each-version-of-the-masterfile-1}

\subparagraph{\texorpdfstring{\emph{Datasets plots with grazed = NA \&
Grazing (1/0) ==
`?':}}{Datasets plots with grazed = NA \& Grazing (1/0) == ?:}}\label{datasets-plots-with-grazed-na-grazing-10}

The following plots in PL\_A have grazed = NA and Grazing (1/0) == `?'

\begin{verbatim}
## [1] "PL_A_N003_0.0001a" "PL_A_N003_0.001a"  "PL_A_N003_0.01a"  
## [4] "PL_A_N003_0.1a"    "PL_A_N003_10a"     "PL_A_N003_1a"
\end{verbatim}

No additional information could be found, so the plots are left as
grazed = NA

\subparagraph{\texorpdfstring{\emph{Datasets plots with grazed == 0 \&
Grazing (1/0) ==
`probably':}}{Datasets plots with grazed == 0 \& Grazing (1/0) == probably:}}\label{datasets-plots-with-grazed-0-grazing-10-probably}

The following plots in PL\_A have grazed == 0 and Grazing (1/0) ==
`'probably'

\begin{verbatim}
## [1] "PL_A_N004_0.0001a" "PL_A_N004_0.001a"  "PL_A_N004_0.01a"  
## [4] "PL_A_N004_0.1a"    "PL_A_N004_10a"     "PL_A_N004_1a"
\end{verbatim}

No additional information could be found, so the plots are left as
grazed == 0.

\subparagraph{\texorpdfstring{\emph{Datasets plots with grazed == 0 \&
Grazing (1/0) ==
`1':}}{Datasets plots with grazed == 0 \& Grazing (1/0) == 1:}}\label{datasets-plots-with-grazed-0-grazing-10-1}

The following plots in PL\_A have grazed == 0 and Grazing (1/0) == 1

\begin{verbatim}
##   [1] "PL_A_N005_0.0001b" "PL_A_N005_0.001b"  "PL_A_N005_0.01b"  
##   [4] "PL_A_N005_0.1b"    "PL_A_N005_10b"     "PL_A_N005_1b"     
##   [7] "UA_L_N001_0.0001a" "UA_L_N001_0.0001b" "UA_L_N001_0.001a" 
##  [10] "UA_L_N001_0.001b"  "UA_L_N001_0.01a"   "UA_L_N001_0.01b"  
##  [13] "UA_L_N001_0.1a"    "UA_L_N001_0.1b"    "UA_L_N001_100"    
##  [16] "UA_L_N001_10a"     "UA_L_N001_10b"     "UA_L_N001_1a"     
##  [19] "UA_L_N001_1b"      "UA_L_N002_0.0001a" "UA_L_N002_0.0001b"
##  [22] "UA_L_N002_0.001a"  "UA_L_N002_0.001b"  "UA_L_N002_0.01a"  
##  [25] "UA_L_N002_0.01b"   "UA_L_N002_0.1a"    "UA_L_N002_0.1b"   
##  [28] "UA_L_N002_100"     "UA_L_N002_10a"     "UA_L_N002_10b"    
##  [31] "UA_L_N002_1a"      "UA_L_N002_1b"      "UA_L_N003_0.0001a"
##  [34] "UA_L_N003_0.0001b" "UA_L_N003_0.001a"  "UA_L_N003_0.001b" 
##  [37] "UA_L_N003_0.01a"   "UA_L_N003_0.01b"   "UA_L_N003_0.1a"   
##  [40] "UA_L_N003_0.1b"    "UA_L_N003_100"     "UA_L_N003_10a"    
##  [43] "UA_L_N003_10b"     "UA_L_N003_1a"      "UA_L_N003_1b"     
##  [46] "UA_L_N004_0.0001a" "UA_L_N004_0.0001b" "UA_L_N004_0.001a" 
##  [49] "UA_L_N004_0.001b"  "UA_L_N004_0.01a"   "UA_L_N004_0.01b"  
##  [52] "UA_L_N004_0.1a"    "UA_L_N004_0.1b"    "UA_L_N004_100"    
##  [55] "UA_L_N004_10a"     "UA_L_N004_10b"     "UA_L_N004_1a"     
##  [58] "UA_L_N004_1b"      "UA_L_N005_0.0001a" "UA_L_N005_0.0001b"
##  [61] "UA_L_N005_0.001a"  "UA_L_N005_0.001b"  "UA_L_N005_0.01a"  
##  [64] "UA_L_N005_0.01b"   "UA_L_N005_0.1a"    "UA_L_N005_0.1b"   
##  [67] "UA_L_N005_100"     "UA_L_N005_10a"     "UA_L_N005_10b"    
##  [70] "UA_L_N005_1a"      "UA_L_N005_1b"      "UA_L_N006_0.0001a"
##  [73] "UA_L_N006_0.0001b" "UA_L_N006_0.001a"  "UA_L_N006_0.001b" 
##  [76] "UA_L_N006_0.01a"   "UA_L_N006_0.01b"   "UA_L_N006_0.1a"   
##  [79] "UA_L_N006_0.1b"    "UA_L_N006_100"     "UA_L_N006_10a"    
##  [82] "UA_L_N006_10b"     "UA_L_N006_1a"      "UA_L_N006_1b"     
##  [85] "UA_L_N007_0.0001a" "UA_L_N007_0.0001b" "UA_L_N007_0.001a" 
##  [88] "UA_L_N007_0.001b"  "UA_L_N007_0.01a"   "UA_L_N007_0.01b"  
##  [91] "UA_L_N007_0.1a"    "UA_L_N007_0.1b"    "UA_L_N007_100"    
##  [94] "UA_L_N007_10a"     "UA_L_N007_10b"     "UA_L_N007_1a"     
##  [97] "UA_L_N007_1b"      "UA_L_N008_0.0001a" "UA_L_N008_0.0001b"
## [100] "UA_L_N008_0.001a"  "UA_L_N008_0.001b"  "UA_L_N008_0.01a"  
## [103] "UA_L_N008_0.01b"   "UA_L_N008_0.1a"    "UA_L_N008_0.1b"   
## [106] "UA_L_N008_100"     "UA_L_N008_10a"     "UA_L_N008_10b"    
## [109] "UA_L_N008_1a"      "UA_L_N008_1b"      "UA_L_N009_0.0001a"
## [112] "UA_L_N009_0.0001b" "UA_L_N009_0.001a"  "UA_L_N009_0.001b" 
## [115] "UA_L_N009_0.01a"   "UA_L_N009_0.01b"   "UA_L_N009_0.1a"   
## [118] "UA_L_N009_0.1b"    "UA_L_N009_100"     "UA_L_N009_10a"    
## [121] "UA_L_N009_10b"     "UA_L_N009_1a"      "UA_L_N009_1b"     
## [124] "UA_L_N010_0.0001a" "UA_L_N010_0.0001b" "UA_L_N010_0.001a" 
## [127] "UA_L_N010_0.001b"  "UA_L_N010_0.01a"   "UA_L_N010_0.01b"  
## [130] "UA_L_N010_0.1a"    "UA_L_N010_0.1b"    "UA_L_N010_100"    
## [133] "UA_L_N010_10a"     "UA_L_N010_10b"     "UA_L_N010_1a"     
## [136] "UA_L_N010_1b"
\end{verbatim}

\begin{itemize}
\tightlist
\item
  PL\_A contains 6 plots (PL\_A\_N005\_xxxxb) noted as abandoned with
  `occasionally grazed or trampled' in the detailed column (column 96).
  These plots will be classified as grazed in the new binary column.
\end{itemize}

\begin{Shaded}
\begin{Highlighting}[]
\NormalTok{df[df}\OperatorTok{$}\StringTok{`}\DataTypeTok{grazed}\StringTok{`} \OperatorTok\StringTok{ '0'} \OperatorTok{&}\StringTok{ }\NormalTok{df}\OperatorTok{$}\StringTok{`}\DataTypeTok{Grazing (1/0)}\StringTok{`} \OperatorTok\StringTok{ '1'}\NormalTok{,]}\OperatorTok\StringTok{ }
\StringTok{    }\KeywordTok{mutate}\NormalTok{(}\DataTypeTok{grazed =} \DecValTok{1}\NormalTok{)}
\end{Highlighting}
\end{Shaded}

\paragraph{Check}\label{check-1}

\begin{table}[H]

\caption{\label{tab:unnamed-chunk-28}Contingency table of columns grazed and Grazing (1/0) after reclassification (check}
\centering
\begin{tabular}[t]{l|r|r|r}
\hline
Grazing (1/0) & 0 & 1 & <NA>\\
\hline
? & NA & 7 & 6\\
\hline
0 & 3585 & 13 & 17\\
\hline
1 & NA & 6949 & NA\\
\hline
2 & NA & 1 & NA\\
\hline
probably & 6 & NA & NA\\
\hline
NA & 124921 & 29290 & 23178\\
\hline
\end{tabular}
\end{table}

\subsection{Burning}\label{burning}

\begin{table}[H]

\caption{\label{tab:unnamed-chunk-29}Contingency table of columns burnt and Burning (1/0)}
\centering
\begin{tabular}[t]{l|r|r}
\hline
Burning (1/0) & 0 & <NA>\\
\hline
? & 6 & NA\\
\hline
0 & 7542 & 62\\
\hline
1 & 187 & 3\\
\hline
x & NA & 34\\
\hline
NA & 157037 & 23102\\
\hline
\end{tabular}
\end{table}

\subsubsection{Datasets with discrepancies between burnt and Burning
(1/0) (automatic
report):}\label{datasets-with-discrepancies-between-burnt-and-burning-10-automatic-report}

The dataset(s) AT\_E contain(s) 34 plots with burnt = NA and Burning
(1/0) = x

The dataset(s) BY\_A, CH\_E, IR\_A, IT\_L, RS\_A, RU\_J, RU\_L, TJ\_A,
UA\_G, UA\_I, UA\_J, UA\_L contain(s) 187 plots with burnt == 0 and
Burning (1/0) = 1

The dataset(s) BY\_A contain(s) 3 plots with burnt = NA and Burning
(1/0) = 1

The dataset(s) PL\_A contain(s) 6 plots with burnt == 0 and Burning
(1/0) = ?

\paragraph{Manual verification of discrepancies (has to be updated
manually if necessary for each version of the
masterfile)}\label{manual-verification-of-discrepancies-has-to-be-updated-manually-if-necessary-for-each-version-of-the-masterfile-2}

\subparagraph{\texorpdfstring{\emph{Datasets with burnt == 0 and Burning
(1/0) ==
1}}{Datasets with burnt == 0 and Burning (1/0) == 1}}\label{datasets-with-burnt-0-and-burning-10-1}

The following plots have burnt = 0 and Burning (1/0) == `1':

\begin{verbatim}
##   [1] "BY_A_P003"         "BY_A_P004"         "BY_A_P005"        
##   [4] "CH_E_N003_0.0001b" "CH_E_N003_0.001b"  "CH_E_N003_0.01a"  
##   [7] "CH_E_N003_0.01b"   "CH_E_N003_0.1a"    "CH_E_N003_0.1b"   
##  [10] "CH_E_N003_100"     "CH_E_N003_10a"     "CH_E_N003_10b"    
##  [13] "CH_E_N003_1a"      "CH_E_N003_1b"      "IR_A_N004_0.0001a"
##  [16] "IR_A_N004_0.0001b" "IR_A_N004_0.001a"  "IR_A_N004_0.001b" 
##  [19] "IR_A_N004_0.01a"   "IR_A_N004_0.01b"   "IR_A_N004_0.1a"   
##  [22] "IR_A_N004_0.1b"    "IR_A_N004_1000"    "IR_A_N004_100a"   
##  [25] "IR_A_N004_100b"    "IR_A_N004_10a"     "IR_A_N004_10b"    
##  [28] "IR_A_N004_1a"      "IR_A_N004_1b"      "IR_A_N004_25a"    
##  [31] "IR_A_N004_25b"     "IT_L_N004_0.0001a" "IT_L_N004_0.0001b"
##  [34] "IT_L_N004_0.001a"  "IT_L_N004_0.001b"  "IT_L_N004_0.01a"  
##  [37] "IT_L_N004_0.01b"   "IT_L_N004_0.1a"    "IT_L_N004_0.1b"   
##  [40] "IT_L_N004_100"     "IT_L_N004_10a"     "IT_L_N004_10b"    
##  [43] "IT_L_N004_1a"      "IT_L_N004_1b"      "RS_A_N010_100"    
##  [46] "RS_A_N011_0.0001a" "RS_A_N011_0.0001b" "RS_A_N011_0.001a" 
##  [49] "RS_A_N011_0.001b"  "RS_A_N011_0.01a"   "RS_A_N011_0.01b"  
##  [52] "RS_A_N011_0.1a"    "RS_A_N011_0.1b"    "RS_A_N011_100"    
##  [55] "RS_A_N011_10a"     "RS_A_N011_10b"     "RS_A_N011_1a"     
##  [58] "RS_A_N011_1b"      "RU_J_P020"         "RU_J_P021"        
##  [61] "RU_L_N001_0.0001a" "RU_L_N001_0.0001b" "TJ_A_N007_0.0001a"
##  [64] "TJ_A_N007_0.0001b" "TJ_A_N007_0.001a"  "TJ_A_N007_0.001b" 
##  [67] "TJ_A_N007_0.01a"   "TJ_A_N007_0.01b"   "TJ_A_N007_0.1a"   
##  [70] "TJ_A_N007_0.1b"    "TJ_A_N007_100"     "TJ_A_N007_10a"    
##  [73] "TJ_A_N007_10b"     "TJ_A_N007_1a"      "TJ_A_N007_1b"     
##  [76] "TJ_A_N008_0.0001a" "TJ_A_N008_0.0001b" "TJ_A_N008_0.001a" 
##  [79] "TJ_A_N008_0.001b"  "TJ_A_N008_0.01a"   "TJ_A_N008_0.01b"  
##  [82] "TJ_A_N008_0.1a"    "TJ_A_N008_0.1b"    "TJ_A_N008_100"    
##  [85] "TJ_A_N008_10a"     "TJ_A_N008_10b"     "TJ_A_N008_1a"     
##  [88] "TJ_A_N008_1b"      "TJ_A_N014_0.0001a" "TJ_A_N014_0.0001b"
##  [91] "TJ_A_N014_0.001a"  "TJ_A_N014_0.001b"  "TJ_A_N014_0.01a"  
##  [94] "TJ_A_N014_0.01b"   "TJ_A_N014_0.1a"    "TJ_A_N014_0.1b"   
##  [97] "TJ_A_N014_100"     "TJ_A_N014_10a"     "TJ_A_N014_10b"    
## [100] "TJ_A_N014_1a"      "TJ_A_N014_1b"      "UA_G_N008_100"    
## [103] "UA_G_N009_1"       "UA_G_N009_10"      "UA_G_N009_100"    
## [106] "UA_G_N010_1"       "UA_G_N010_10"      "UA_G_N010_100"    
## [109] "UA_G_N011_1"       "UA_G_N011_10"      "UA_G_N011_100"    
## [112] "UA_G_N012_1"       "UA_G_N012_10"      "UA_G_N012_100"    
## [115] "UA_I_N005_0.001a"  "UA_I_N005_0.001b"  "UA_I_N005_0.01a"  
## [118] "UA_I_N005_0.01b"   "UA_I_N005_0.1a"    "UA_I_N005_0.1b"   
## [121] "UA_I_N005_100"     "UA_I_N005_10a"     "UA_I_N005_10b"    
## [124] "UA_I_N005_1a"      "UA_I_N005_1b"      "UA_I_N006_0.0001a"
## [127] "UA_I_N006_0.0001b" "UA_J_N008_0.0001a" "UA_J_N008_0.0001b"
## [130] "UA_J_N008_0.001a"  "UA_J_N008_0.001b"  "UA_J_N008_0.01a"  
## [133] "UA_J_N008_0.01b"   "UA_J_N008_0.1a"    "UA_J_N008_0.1b"   
## [136] "UA_J_N008_100"     "UA_J_N008_10a"     "UA_J_N008_10b"    
## [139] "UA_J_N008_1a"      "UA_J_N008_1b"      "UA_J_N009_0.0001a"
## [142] "UA_J_N009_0.0001b" "UA_J_N009_0.001a"  "UA_J_N009_0.001b" 
## [145] "UA_J_N009_0.01a"   "UA_J_N009_0.01b"   "UA_J_N009_0.1a"   
## [148] "UA_J_N009_0.1b"    "UA_J_N009_100"     "UA_J_N009_10a"    
## [151] "UA_J_N009_10b"     "UA_J_N009_1a"      "UA_J_N009_1b"     
## [154] "UA_J_P010"         "UA_J_P017"         "UA_J_P052"        
## [157] "UA_J_P053"         "UA_J_P074"         "UA_J_P076"        
## [160] "UA_J_P081"         "UA_J_P095"         "UA_L_N011_0.0001a"
## [163] "UA_L_N011_0.0001b" "UA_L_N011_0.001a"  "UA_L_N011_0.001b" 
## [166] "UA_L_N011_0.01a"   "UA_L_N011_0.01b"   "UA_L_N011_0.1a"   
## [169] "UA_L_N011_0.1b"    "UA_L_N011_100"     "UA_L_N011_10a"    
## [172] "UA_L_N011_10b"     "UA_L_N011_1a"      "UA_L_N011_1b"     
## [175] "UA_L_N012_0.0001a" "UA_L_N012_0.0001b" "UA_L_N012_0.001a" 
## [178] "UA_L_N012_0.001b"  "UA_L_N012_0.01a"   "UA_L_N012_0.01b"  
## [181] "UA_L_N012_0.1a"    "UA_L_N012_0.1b"    "UA_L_N012_100"    
## [184] "UA_L_N012_10a"     "UA_L_N012_10b"     "UA_L_N012_1a"     
## [187] "UA_L_N012_1b"
\end{verbatim}

No other information are found. These plots are reclassified with burnt
== 1

\begin{Shaded}
\begin{Highlighting}[]
\NormalTok{df[df}\OperatorTok{$}\NormalTok{burnt }\OperatorTok\StringTok{ }\DecValTok{0} \OperatorTok{&}\StringTok{ }\NormalTok{df}\OperatorTok{$}\StringTok{`}\DataTypeTok{Burning (1/0)}\StringTok{`} \OperatorTok\StringTok{ '1'}\NormalTok{,]}\OperatorTok
\StringTok{    }\KeywordTok{mutate}\NormalTok{(}\DataTypeTok{burnt =} \DecValTok{1}\NormalTok{)}
\end{Highlighting}
\end{Shaded}

\subparagraph{\texorpdfstring{\emph{Datasets with burnt = NA and Burning
(1/0) ==
1}}{Datasets with burnt = NA and Burning (1/0) == 1}}\label{datasets-with-burnt-na-and-burning-10-1}

The following plots have burnt = NA and Burning (1/0) == `?':

\begin{verbatim}
## [1] "BY_A_P006" "BY_A_P007" "BY_A_P009"
\end{verbatim}

No other information are found. These plots are reclassified with burnt
== 1

\begin{Shaded}
\begin{Highlighting}[]
\NormalTok{df[}\KeywordTok{is.na}\NormalTok{(df}\OperatorTok{$}\NormalTok{burnt) }\OperatorTok{&}\StringTok{ }\NormalTok{df}\OperatorTok{$}\StringTok{`}\DataTypeTok{Burning (1/0)}\StringTok{`} \OperatorTok\StringTok{ '1'}\NormalTok{,]}\OperatorTok
\StringTok{    }\KeywordTok{mutate}\NormalTok{(}\DataTypeTok{burnt =} \DecValTok{1}\NormalTok{)}
\end{Highlighting}
\end{Shaded}

\subparagraph{\texorpdfstring{\emph{Datasets with burnt = NA and Burning
(1/0) ==
?}}{Datasets with burnt = NA and Burning (1/0) == ?}}\label{datasets-with-burnt-na-and-burning-10}

The following plots have burnt == 0 and Burning (1/0) == `?':

\begin{verbatim}
## [1] "PL_A_N005_0.0001a" "PL_A_N005_0.001a"  "PL_A_N005_0.01a"  
## [4] "PL_A_N005_0.1a"    "PL_A_N005_10a"     "PL_A_N005_1a"
\end{verbatim}

No additional information could be found, so the plots are left as burnt
== 0.

\paragraph{Check}\label{check-2}

\begin{table}[H]

\caption{\label{tab:unnamed-chunk-36}Contingency table of columns burnt and Burning (1/0) after reclassification (check)}
\centering
\begin{tabular}[t]{l|r|r|r}
\hline
Burning (1/0) & 0 & 1 & <NA>\\
\hline
? & 6 & NA & NA\\
\hline
0 & 7542 & NA & 62\\
\hline
1 & NA & 190 & NA\\
\hline
x & NA & NA & 34\\
\hline
NA & 157037 & NA & 23102\\
\hline
\end{tabular}
\end{table}

\subsection{Fertilization}\label{fertilization}

Non applicable so far, since the old binary column was an intensity one.

\section{Former land-use.}\label{former-land-use.}

\subsubsection{\texorpdfstring{\emph{Datasets containing plots for which
years\_abandonment \textgreater{} 0 \& `Ex arable (1/0)' !=
1:}}{Datasets containing plots for which years\_abandonment \textgreater{} 0 \& Ex arable (1/0) != 1:}}\label{datasets-containing-plots-for-which-years_abandonment-0-ex-arable-10-1}

\begin{itemize}
\tightlist
\item
  The datasets contain 0 plots having an abandonment duration but
  classified as Ex arable == 0.
\end{itemize}

So far, the only former land-use (and years since abandonment)
information is related to arable field. As we aim at creating a
gathering information about the time since abandonment for every former
land-use (arable, mown, grazed, burnt, recreational), we create a new
text column \texttt{former\_land\_use}. Right now, this column only
contains \texttt{arable}, but will be populated later.

\begin{Shaded}
\begin{Highlighting}[]
\NormalTok{df <-}\StringTok{ }\NormalTok{df}\OperatorTok
\StringTok{    }\KeywordTok{mutate}\NormalTok{(}\DataTypeTok{former_land_use =} \KeywordTok{ifelse}\NormalTok{(}\StringTok{`}\DataTypeTok{Ex arable (1/0)}\StringTok{`} \OperatorTok\StringTok{ }\DecValTok{1}\NormalTok{, }\StringTok{'arable'}\NormalTok{, }\OtherTok{NA}\NormalTok{))}
\end{Highlighting}
\end{Shaded}

\textbf{Post creation check} number of plots with former\_land\_use ==
`arable' - number of plots with Ex arable (1/0) == 1 (should be 0): 0

\section{Proposal for a new land-use
template}\label{proposal-for-a-new-land-use-template}

The column 96 provide a lot of information that is so far not fully
exploited. To make use of it, it requires to manually check each of the
value in this column, to fill other columns. After the Skype meeting
(26th), I propose the following columns types:

\subsection{Land-use}\label{land-use}

Already explained above, these are binary columns providing information
on the land-use There is one extra column (other\_lu) that contains
verbal information not found in these five columns.

\begin{itemize}
\tightlist
\item
  mown
\item
  grazed
\item
  burnt
\item
  fertilized
\item
  abandoned
\item
  natural
\item
  other\_lu
\end{itemize}

\subsection{Land-use intensity}\label{land-use-intensity}

\subsubsection{Mowing}\label{mowing-1}

\begin{itemize}
\tightlist
\item
  mowing\_frequency. Numeric, from 0 to infinite. Number of cut(s) per
  year
\end{itemize}

\subsubsection{Grazing}\label{grazing-1}

\begin{itemize}
\tightlist
\item
  grazing\_intensity. Numeric, from 0 to 1. Self reported relative
  intensity of grazing.
\item
  grazing\_load. Numeric, from 0 to infinite. Grazing animal
  load/ha./year (maybe expressed as equivalent cattle)
\end{itemize}

\subsubsection{Burning}\label{burning-1}

\begin{itemize}
\tightlist
\item
  burning\_frequency. Numeric, from 0 to infinite. Number of burning(s)
  per year
\end{itemize}

\subsubsection{Fertilization}\label{fertilization-1}

\begin{itemize}
\tightlist
\item
  fertilization\_intensity. Numeric, from 0 to 1. Self reported relative
  intensity of fertilization
\item
  fertilization\_type. Text, `synthetic' or `natural'.
\item
  fertilization\_details. Text, free values. Details not present in the
  tzo previous column (\textbf{e.g.} frequency, amount, type of chemical
  fertilizer\ldots{})
\end{itemize}

\subsection{Abandonement}\label{abandonement}

\begin{itemize}
\tightlist
\item
  years\_abandonment. Numeric, from 0 to infinity. Time since
  abandonment in years
\item
  former\_land\_use: Text, 5 values that can be combined (arable, mown,
  grazed, burnt, recreational). Land-use before abandonment.
\end{itemize}

\subsection{Grazing details}\label{grazing-details}

\begin{itemize}
\tightlist
\item
  grazing\_animal: Text, free values that can be combined, spearated by
  `/'. New values should be added in agreement with the DB manager.Type
  of animal grazing the land.
\end{itemize}

\subsection{Land destination}\label{land-destination}

\begin{itemize}
\tightlist
\item
  land\_destination: Text, 3 possible values (cropland, farmland,
  recreational). The primary intent behind the land-use.
\end{itemize}

\subsection{Grassland type}\label{grassland-type}

\begin{itemize}
\tightlist
\item
  primary\_grassland: Numeric, 0 or 1. Is the grassland primary (1) or
  secondary (0)?
\end{itemize}

\begin{longtable}[]{@{}llll@{}}
\caption{Summary of the proposed template (\textbf{to be
discussed})}\tabularnewline
\toprule
\begin{minipage}[b]{0.15\columnwidth}\raggedright\strut
Column type\strut
\end{minipage} & \begin{minipage}[b]{0.30\columnwidth}\raggedright\strut
Column name\strut
\end{minipage} & \begin{minipage}[b]{0.16\columnwidth}\raggedright\strut
Variable type\strut
\end{minipage} & \begin{minipage}[b]{0.28\columnwidth}\raggedright\strut
Possible values\strut
\end{minipage}\tabularnewline
\midrule
\endfirsthead
\toprule
\begin{minipage}[b]{0.15\columnwidth}\raggedright\strut
Column type\strut
\end{minipage} & \begin{minipage}[b]{0.30\columnwidth}\raggedright\strut
Column name\strut
\end{minipage} & \begin{minipage}[b]{0.16\columnwidth}\raggedright\strut
Variable type\strut
\end{minipage} & \begin{minipage}[b]{0.28\columnwidth}\raggedright\strut
Possible values\strut
\end{minipage}\tabularnewline
\midrule
\endhead
\begin{minipage}[t]{0.24\columnwidth}\raggedright\strut
Land use\strut
\end{minipage} & \begin{minipage}[t]{0.24\columnwidth}\raggedright\strut
\begin{itemize}
\tightlist
\item
  mown
\item
  grazed
\item
  burnt
\item
  fertilized
\item
  abandoned
\item
  natural
\item
  other\_lu
\end{itemize}\strut
\end{minipage} & \begin{minipage}[t]{0.24\columnwidth}\raggedright\strut
\begin{itemize}
\tightlist
\item
  binary
\item
  binary
\item
  binary
\item
  binary
\item
  binary
\item
  binary
\item
  text
\end{itemize}\strut
\end{minipage} & \begin{minipage}[t]{0.24\columnwidth}\raggedright\strut
\begin{itemize}
\tightlist
\item
  0/1
\item
  0/1
\item
  0/1
\item
  0/1
\item
  0/1
\item
  0/1
\item
  free
\end{itemize}\strut
\end{minipage}\tabularnewline
\begin{minipage}[t]{0.24\columnwidth}\raggedright\strut
Land-use intensity\strut
\end{minipage} & \begin{minipage}[t]{0.24\columnwidth}\raggedright\strut
\begin{itemize}
\tightlist
\item
  grazing\_intensity
\item
  grazing\_load
\item
  mowing\_frequency
\item
  burning\_frequency
\item
  fertilization\_intensity
\item
  fertilization\_type
\item
  fertilization\_details
\end{itemize}\strut
\end{minipage} & \begin{minipage}[t]{0.24\columnwidth}\raggedright\strut
\begin{itemize}
\tightlist
\item
  numeric
\item
  numeric
\item
  numeric
\item
  numeric
\item
  text
\item
  numeric
\item
  text
\end{itemize}\strut
\end{minipage} & \begin{minipage}[t]{0.24\columnwidth}\raggedright\strut
\begin{itemize}
\tightlist
\item
  0 to 1
\item
  0 to infinity
\item
  0 to infinity
\item
  0 to infinity
\item
  0 to 1
\item
  synthetic/natural
\item
  free
\end{itemize}\strut
\end{minipage}\tabularnewline
\begin{minipage}[t]{0.24\columnwidth}\raggedright\strut
Abandonment\strut
\end{minipage} & \begin{minipage}[t]{0.24\columnwidth}\raggedright\strut
\begin{itemize}
\tightlist
\item
  years\_abandonment
\item
  former\_land\_use
\end{itemize}\strut
\end{minipage} & \begin{minipage}[t]{0.24\columnwidth}\raggedright\strut
\begin{itemize}
\tightlist
\item
  numeric
\item
  text
\end{itemize}\strut
\end{minipage} & \begin{minipage}[t]{0.24\columnwidth}\raggedright\strut
\begin{itemize}
\tightlist
\item
  0 to infinity
\item
  arable, mown, grazed burnt, recreational
\end{itemize}\strut
\end{minipage}\tabularnewline
\begin{minipage}[t]{0.24\columnwidth}\raggedright\strut
Grazing details\strut
\end{minipage} & \begin{minipage}[t]{0.24\columnwidth}\raggedright\strut
\begin{itemize}
\tightlist
\item
  grazing animal
\end{itemize}\strut
\end{minipage} & \begin{minipage}[t]{0.24\columnwidth}\raggedright\strut
\begin{itemize}
\tightlist
\item
  text
\end{itemize}\strut
\end{minipage} & \begin{minipage}[t]{0.24\columnwidth}\raggedright\strut
\begin{itemize}
\tightlist
\item
  free
\end{itemize}\strut
\end{minipage}\tabularnewline
\begin{minipage}[t]{0.24\columnwidth}\raggedright\strut
Land destination\strut
\end{minipage} & \begin{minipage}[t]{0.24\columnwidth}\raggedright\strut
\begin{itemize}
\tightlist
\item
  land\_destination
\end{itemize}\strut
\end{minipage} & \begin{minipage}[t]{0.24\columnwidth}\raggedright\strut
\begin{itemize}
\tightlist
\item
  text
\end{itemize}\strut
\end{minipage} & \begin{minipage}[t]{0.24\columnwidth}\raggedright\strut
\begin{itemize}
\tightlist
\item
  cropland, farmland, recreational
\end{itemize}\strut
\end{minipage}\tabularnewline
\begin{minipage}[t]{0.24\columnwidth}\raggedright\strut
Grassland type\strut
\end{minipage} & \begin{minipage}[t]{0.24\columnwidth}\raggedright\strut
\begin{itemize}
\tightlist
\item
  primary\_grassland
\end{itemize}\strut
\end{minipage} & \begin{minipage}[t]{0.24\columnwidth}\raggedright\strut
\begin{itemize}
\tightlist
\item
  numeric
\end{itemize}\strut
\end{minipage} & \begin{minipage}[t]{0.24\columnwidth}\raggedright\strut
\begin{itemize}
\tightlist
\item
  0/1
\end{itemize}\strut
\end{minipage}\tabularnewline
\bottomrule
\end{longtable}

\subsection{Hamonizing the land-use details
column}\label{hamonizing-the-land-use-details-column}

The following adds the columns not existing yet in the DB, and export
the template, with one row per unique value of the land-use detail
column.\\
This exported file is to be checked manually, in order to fill in the
template. To help the sorting, the plots and/or dataset name is left.\\
In a second step, it will be reintegrated to the DB, to finalize the
harmonization

\begin{Shaded}
\begin{Highlighting}[]
\KeywordTok{names}\NormalTok{(df)[}\KeywordTok{which}\NormalTok{(}\KeywordTok{names}\NormalTok{(df) }\OperatorTok{==}\StringTok{ 'Land use detail (e.g. number of cuts or years since abandonment)'}\NormalTok{)] <-}\StringTok{ "land_use_details"}

\NormalTok{template_names <-}\StringTok{ }\KeywordTok{c}\NormalTok{(}\StringTok{'mown'}\NormalTok{,}\StringTok{'grazed'}\NormalTok{,}\StringTok{'burnt'}\NormalTok{,}\StringTok{'fertilized'}\NormalTok{,}\StringTok{'abandoned'}\NormalTok{,}
                    \StringTok{'natural'}\NormalTok{,}\StringTok{'other_lu'}\NormalTok{,}\StringTok{'grazing_intensity'}\NormalTok{,}\StringTok{'grazing_load'}\NormalTok{,}
                    \StringTok{'mowing_frequency'}\NormalTok{,}\StringTok{'burning_frequency'}\NormalTok{,}
                    \StringTok{'fertilization_intensity'}\NormalTok{,}\StringTok{'fertilization_type'}\NormalTok{,}
                    \StringTok{'fertilization_details'}\NormalTok{,}\StringTok{'years_abandonment'}\NormalTok{,}
                    \StringTok{'former_land_use'}\NormalTok{,}\StringTok{'grazing animal'}\NormalTok{,}\StringTok{'land_destination'}\NormalTok{,}
                    \StringTok{'primary_grassland'}\NormalTok{)}

\NormalTok{df_details <-}\StringTok{ }\NormalTok{df[}\KeywordTok{match}\NormalTok{(}\KeywordTok{unique}\NormalTok{(df}\OperatorTok{$}\NormalTok{land_use_details), df}\OperatorTok{$}\NormalTok{land_use_details),]}

\ControlFlowTok{for}\NormalTok{(colname }\ControlFlowTok{in}\NormalTok{ template_names)\{}
  \ControlFlowTok{if}\NormalTok{(}\OperatorTok{!}\NormalTok{colname }\OperatorTok\StringTok{ }\KeywordTok{names}\NormalTok{(df_details))}
\NormalTok{    df_details[,colname] <-}\StringTok{ }\OtherTok{NA}
\NormalTok{\}}

\NormalTok{df_details <-}\StringTok{ }\NormalTok{df_details[,}\KeywordTok{c}\NormalTok{(}\StringTok{'GrassPlot ID of plot'}\NormalTok{, }\StringTok{'land_use_details'}\NormalTok{, template_names)]}


\KeywordTok{write_xlsx}\NormalTok{(df_details, }\DataTypeTok{path =} \KeywordTok{file.path}\NormalTok{(path_grassplot, }\StringTok{'df_details_v1.20.xlsx'}\NormalTok{))}
\end{Highlighting}
\end{Shaded}


\end{document}
